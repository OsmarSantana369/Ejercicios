\documentclass[12pt, fleqn]{article}
\usepackage[spanish]{babel}
\usepackage[utf8]{inputenc}
\usepackage[margin = 15mm, left = 12mm]{geometry}
\usepackage[none]{hyphenat}
\usepackage{amsmath, amssymb, amsfonts}
\usepackage{parskip}

\expandafter\def\expandafter\normalsize\expandafter{%
    \setlength\abovedisplayskip{-9pt}%
    \setlength\belowdisplayskip{5pt}%
}

\newcommand{\complejos}{\mathbb{C}}
\newcommand{\cis}[2]{\cos \left( #1 \right) #2 i \sen \left( #1 \right)}
\newcommand{\clase}[1]{$ \mathcal{C}^#1 $}

\begin{document}
	\begin{enumerate}
%%%%%%%%%%%%%%%%%%%%%%%%%%%%%%%%%%%%%%%%%%%%%%%%%% Ejercicio 1 %%%%%%%%%%%%%%%%%%%%%%%%%%%%%%%%%%%%%%%%%%%%%%%%%%%%%%%%
		\item \begin{enumerate}
			\item Solución.
			
			Se tiene que:

			\begin{align*}
				z^3 - 27i = 0 \Longleftrightarrow & z^3 = 27i \\
				%
				\Longleftrightarrow & z = \sqrt[3]{27} \left[ \cis{\dfrac{\dfrac{\pi}{2}}{3}}{+} \right] \mbox{ ó } \\
				& z = \sqrt[3]{27} \left[ \cis{\dfrac{\dfrac{\pi}{2} + 2 \pi}{3}}{+} \right] \mbox{ ó } \\
				& z = \sqrt[3]{27} \left[ \cis{\dfrac{\dfrac{\pi}{2} + 4 \pi}{3}}{+} \right] \\
				%
				\Longleftrightarrow & z = 3 \left[ \cis{\dfrac{\pi}{6}}{+} \right] \mbox{ ó } \\
				& z = 3 \left[ \cis{\dfrac{5 \pi}{6}}{+} \right] \mbox{ ó } \\
				& z = 3 \left[ \cis{\dfrac{9 \pi}{6}}{+} \right] \\
				%
				\Longleftrightarrow & z = \dfrac{3 \sqrt{3}}{2} + i \dfrac{3}{2} \mbox{ ó } \\
				& z = - \dfrac{3 \sqrt{3}}{2} + i \dfrac{3}{2} \mbox{ ó } \\
				& z = -3i \\
			\end{align*}
			
			Ya que las funciones racionales son holomorfas en $ \complejos $, excepto en los valores donde se anula el denominador, por lo anterior, se concluye que $f$ es holomorfa en $ \complejos \setminus \left\lbrace \frac{3 \sqrt{3}}{2} + i \frac{3}{2}, - \frac{3 \sqrt{3}}{2} + i \frac{3}{2}, -3i \right\rbrace $. Además,

			\begin{align*}
				f'(z) &= \dfrac{(z^3 - 27i)(12z^3 - 4z) - (3z^4 - 2z^2 + i) (3z^2)}{(z^3 - 27i)^2} \\
				%
				&= \dfrac{12z^6 - 324 iz^3 - 4z^4 + 108iz - 9z^6 + 6z^4 - 3iz^2}{(z^3 - 27i)^2} \\
				%
				&= \dfrac{3z^6 - 324 iz^3 + 108iz + 2z^4 - 3iz^2}{(z^3 - 27i)^2} \\
			\end{align*}

			\item Demostración.
			
			Ya que $ f $ es holomorfa, se tiene que $f$ es de clase \clase{\infty}, por lo que, la segunda derivada de $f$ existe, por lo que las segundas derivadas de $u$ y $v$ existen. Luego, por las ecuaciones de Cauchy-Riemann, se obtiene que

			\begin{align*}
				u_x = v_y & \Longrightarrow u_{xx} = v_{yx} = v_{xy}, \\
				%
				& \Longrightarrow u_{xy} = v_{yy}, \\
				%
				u_y = -v_x & \Longrightarrow u_{yy} = -v_{xy} \mbox{ y }, \\
				%
				& \Longrightarrow u_{xy} = u_{yx} = -v_{xx},
			\end{align*}

			De esta forma, para todo $ z \in A $, se da que

			\begin{align*}
				u_{xx} (z) = -u_{yy} (z) \quad \mbox{y} \quad v_{yy} (z) = -v_{xx} (z) \\
				%
				\Longrightarrow u_{xx} (z) + u_{yy} (z) = 0 \quad \mbox{y} \quad v_{xx} (z) + v_{yy} (z) = 0
			\end{align*}
			
			Por lo tanto, $u$ y $v$ son armónicas.
		\end{enumerate}
%%%%%%%%%%%%%%%%%%%%%%%%%%%%%%%%%%%%%%%%%%%%%%%%%% Ejercicio 2 %%%%%%%%%%%%%%%%%%%%%%%%%%%%%%%%%%%%%%%%%%%%%%%%%%%%%%%%
		\item \begin{enumerate}
			\item Demostración.
			
			Sea $ z = x + iy \in \complejos $, se tiene que

			\begin{align*}
				f(z) &= \sen (\overline{z}) \\
				%
				&= \dfrac{e^{i\overline{z}} - e^{-i\overline{z}}}{2i} \\
				%
				&= \dfrac{e^{i(x - iy)} - e^{-i(x - iy)}}{2i} \\
				%
				&= \dfrac{e^{y + ix} - e^{-y - ix}}{2i} \\
				%
				&= \dfrac{e^{y} \left( \cis{x}{+} \right) - e^{-y} \left( \cis{x}{-} \right)}{2i} \\
				%
				&= \dfrac{\left( e^{y} - e^{-y} \right) \cos(x)}{2i} + \dfrac{\left( e^{y} + e^{-y} \right) \sen(x)}{2}\\
				%
				&= \dfrac{\left( e^{y} + e^{-y} \right) \sen(x)}{2} + i \dfrac{\left( e^{-y} - e^{y} \right) \cos(x)}{2} \\
			\end{align*}

			Así, $ f(x + iy) = u(x, y) + iv(x,y) $, donde $ u(x,y) = \dfrac{\left( e^{y} + e^{-y} \right) \sen(x)}{2} $ y \newline $ v(x,y) = \dfrac{\left( e^{-y} - e^{y} \right) \cos(x)}{2} $. Estas funciones tienen derivada, por lo que

			\begin{align*}
				u_x &= \dfrac{\left( e^{y} + e^{-y} \right) \cos(x)}{2}, v_y = \dfrac{\left( -e^{-y} - e^{y} \right) \cos(x)}{2}, u_y = \dfrac{\left( e^{y} - e^{-y} \right) \sen(x)}{2} \quad \mbox{y} \\
				v_x &= \dfrac{\left( e^{y} - e^{-y} \right) \sen(x)}{2}
			\end{align*}

			Pero $ u_x \neq v_y $ y $ u_y \neq -v_x $, es decir, $f$ no satisface las ecuaciones de Cauchy-Riemann, por lo que $f$ no es holomorfa en $z$. Por lo tanto, $f$ no es holomorfa en ningun punto del plano complejo, pues $z$ fue arbitrario.

			\item Solución.
			
			Ya que $f$ es una función entera, se da que $ f'(z) = 4z^3 $. De este modo, $ f'(z) = 0 $ si $ z = 0 $, por lo cual $f$ no es conforme en $ z = 0 $, pero si en cualquier otro punto del plano complejo.

			Luego, por una observación, se sabe que las funciones conformes conservan ángulos. Si se considera el segmento de recta que va del origen a $1$, la curva $ C_1 $, y el segmento de recta que va del origen a $i$, la curva $ C_2 $, entonces estas curvas se intersectan en el origen y el ángulo que hay entre ellas es de $ \frac{\pi}{2} $. Después, la imágen de $ C_1 $ y $ C_2 $, bajo $f$, es la curva $ C_1 $, en ambos casos, por lo que las imágenes, bajo $f$, de estas curvas tienen un ángulo de $0$. De esta manera, la propiedad de que los ángulos se preservan, se pierde, dado que $f$ no es conforme en $0$.

			%Por último, sea $ U $ un abierto que contiene al origen, por definición de abierto, existe $ \epsilon > 0 $ tal que $ B_\epsilon (0) \subseteq U $. Si $ z_0 \in B_\epsilon (0) $, entonces existen $ w_0, w_1, w_2, w_3 \in \complejos $ tales que, bajo $f$, se transforman en $ z_0 $, ya que un complejo siempre tiene $n$ raíces $n$-ésimas.
		\end{enumerate}
	\end{enumerate}
\end{document}