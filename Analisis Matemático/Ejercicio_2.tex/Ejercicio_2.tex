\documentclass[fleqn,12pt]{article}
\usepackage[spanish]{babel}
\usepackage[margin = 24mm]{geometry}
\usepackage{amsmath, amssymb, amsfonts}
\usepackage{parskip}

\begin{document}
	\textbf{Proposición.} Sean $ A, B \subseteq \mathbb{R} $ no vacíos. Si $ a \leq b $ para todo $ a \in A $ y para todo $ b \in B $, entonces $ A $ está acotada superiormente y $ B $ está acotado inferiormente, y además, $ \sup (A) \leq \inf (B) $.

	\textbf{Demostración.} 

	Sea $ b \in B $, ya que $ a \leq b $ para todo $ a \in A $, se tiene que $ A $ está acotado superiormente. De igual forma, sea $ a \in A $ como $ a \leq b $ para todo $ b \in B $, se da que $ B $ está acotado inferiormente. Además, dado que $ A $ y $ B $ son no vacíos, se obtiene que el supremo y el ínfimo de $ A $ y $ B $ existen, respectivamente. Como todo elemento de $ B $ es cota superior de $ A $ se tiene que $ \sup (A) $ es una cota inferior de $ B $. Por lo tanto, $ \sup (A) \leq \inf (B) $, pues $ \inf(B) $ es la mayor cota inferior de $ B $.
\end{document}