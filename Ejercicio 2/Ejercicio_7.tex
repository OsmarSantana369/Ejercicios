\documentclass{article}
\usepackage{amsmath}

\begin{document}
      7. $ [\sin (x) \sin (y) - xe^y] \mathrm{d}y = [e^y + \cos (x) \cos (y)] \mathrm{d}x $

      Escribiendo la E.D. en la forma diferencial:

      $ [-e^y - \cos (x) \cos (y)] \mathrm{d}x + [\sin (x) \sin (y) - xe^y] \mathrm{d}y = 0 $

      Sean $ M(x,y) = -e^y - \cos (x) \cos (y) $ y $ N(x,y) = \sin (x) \sin (y) - xe^y $. Luego,

      $ \frac{\partial M}{\partial y} = -e^y + \cos (x) \sin (y) $ y

      $ \frac{\partial N}{\partial x} = \cos (x) \sin (y) -e^y $

      Como $ \frac{\partial M}{\partial y} = \frac{\partial N}{\partial x} $ entonces la E.D. es exacta. Es decir, $ \exists f(x,y) = C $ tal que

      \begin{equation}
            \label{1}
            \frac{\partial f}{\partial x} = -e^y - \cos (x) \cos (y)
      \end{equation}

      \begin{equation}
            \label{2}
            \frac{\partial f}{\partial y} = \sin (x) \sin (y) - xe^y
      \end{equation}

      Integrando ~\ref{1} respecto a $ x $:
      \begin{equation}
            \label{3}
            f(x,y) = -xe^y - \sin (x) \cos (y) + h(y)
      \end{equation}

      Luego, derivando ~\ref{3} respecto a $ y $:

      $ \frac{\partial f}{\partial y} = - xe^y + \sin (x) \sin (y) + h'(y) $

      De la ecuación ~ref{2} se tiene que

      $ \sin (x) \sin (y) - xe^y = - xe^y + \sin (x) \sin (y) + h'(y) $

      $ \Longrightarrow 0 = h'(y) $

      $ \Longrightarrow h(y) = c $

      Sustituyendo en ~\ref{3}:

      $ f(x,y) = -xe^y - \sin (x) \cos (y) + c $

      Como $ f(x,y) = C $ entonces

      $ C = -xe^y - \sin (x) \cos (y) + c $
      
      $ \therefore xe^y + \sin (x) \cos (y) = D ($ con $ D = c - C $) es solución implícita de la E.D.
      
      
      
/end{document}