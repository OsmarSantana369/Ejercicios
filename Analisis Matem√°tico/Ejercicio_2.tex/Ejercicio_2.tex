\documentclass[12pt]{article}
\usepackage[spanish]{babel}
\usepackage[margin = 15mm, top = 12mm]{geometry}
\usepackage{amsmath, amssymb, amsfonts}
\usepackage{parskip}
\usepackage{tcolorbox}       % Agregar cajas con colores
\usepackage{tikz}
\tcbuselibrary{theorems}       % Crear entornos para teoremas
\tcbuselibrary{breakable}       % Para que los entornos de teoremas puedan sobrepasar una página
\usepackage{colortbl}
\usepackage{array, tabularx}       % Insertar tablas en cajas de texto
\usepackage[pdftex, hidelinks]{hyperref}       %Poner al final de los paquetes
\tcbuselibrary{skins}
\usetikzlibrary{shadings}
\usepackage{enumerate}
\usepackage{multirow}
\usepackage{multicol}
\usepackage{graphicx}
\usepackage{setspace}

\usepackage[proportional,scaled=1]{erewhon}
\usepackage[erewhon,vvarbb,bigdelims]{newtxmath}
\usepackage[T1]{fontenc}
\renewcommand*\oldstylenums[1]{\textosf{#1}}

\author{Osmar Dominique Santana Reyes}
\date{\today}

\expandafter\def\expandafter\normalsize\expandafter{%
    \setlength\abovedisplayskip{-10pt}%
    \setlength\belowdisplayskip{7pt}%
}

\tcbset{theorem full label supplement={hypertarget={#1}}}

\newtcbtheorem{defi}{D\hspace{0.1mm}e\hspace{0.1mm}f\hspace{0.4mm}i\hspace{0.1mm}n\hspace{0.1mm}i\hspace{0.1mm}c\hspace{0.1mm}i\hspace{0.1mm}ó\hspace{0.1mm}n}{colback = red!9, colframe = red!72!black, fonttitle = \bfseries, separator sign = {\hspace{2mm}}}{def}

\newtcbtheorem[auto counter]{ejer}{E\hspace{0.1mm}j\hspace{0.1mm}e\hspace{0.1mm}r\hspace{0.1mm}c\hspace{0.1mm}i\hspace{0.1mm}c\hspace{0.1mm}i\hspace{0.1mm}o}{colback = white!98!black, colframe = black!99!white, fonttitle = \bfseries, separator sign = {\hspace{2mm}}}{ejer}

\newenvironment{definicion}[1]{\begin{defi}[breakable, pad at break = 5mm, leftrule = 0.7mm, rightrule = 0.7mm, right = 2mm, left = 2mm, enlarge bottom finally by = 3mm]{}{#1}}{\end{defi}}
\newenvironment{ejercicio}[1]{\begin{ejer}[breakable, pad at break = 5mm, leftrule = 0.7mm, rightrule = 0.7mm, right = 2mm, left = 2mm, enlarge bottom finally by = 3mm, fontlower = \setlength{\parskip}{2mm}]{}{#1}}{\end{ejer}}

\newcommand{\paratodo}{\, \forall \,}
\newcommand{\existe}{\exists \,}
\newcommand{\talque}{\; \mathbf{\colon}}
\newcommand{\rsi}[1]{\mathcal{R}(#1)}
\newcommand{\nat}{\mathbb{N}}
\newcommand{\real}{\mathbb{R}}
\newcommand{\intg}[4]{\int_{#1}^{#2} #3 \, \mathrm{d} #4}

\begin{document}
	\begin{ejercicio}{diferenciabilidad}
		Sean $ f: U \subseteq \real^n \to \real^m $, con $ U $ un abierto de $ \real^n $. Si existen las derivadas parciales de $ f $ y como funciones son continuas en $ U $ entonces $ f $ es diferenciable.

		\tcblower

		\textbf{Demostración.} 

		Sean $ \overline{y} = \left( y_1, y_2, \ldots, y_n \right) \in \real^n $ fijo y $ f = \left( f_1, f_2, \ldots, f_n \right) $.

		P.d. $ \displaystyle \lim_{\overline{x} \to \overline{y}} \dfrac{ \phantom{|} \left\lvert f_i(\overline{x}) - f_i(\overline{y}) - \displaystyle \sum_{j=1}^{n} \dfrac{ \partial f_i }{ \partial x_j } (\overline{x}) \left( x_j - y_j \right) \right\rvert \phantom{|} }{ \left\lVert \overline{x} - \overline{y} \right\rVert } = 0 \quad \forall \, i = 1, 2, \ldots, m $.

		Para todo $ i = 1, 2, \ldots, m $ \, y \, $ j = 1, 2, \ldots, n $, sea $ g : \real \to \real $ dada por \\ $ g_j(x) = f_i\left( x_1, \ldots, x_{j-1}, x, y_{j+1}, \ldots, y_n \right) $, donde $ \overline{x} = \left( x_1, x_2, \ldots, x_n \right) \in U $. Luego, para cada \\ $ i = 1, 2, \ldots, m $, se tiene que 

		\begin{equation*}
			\begin{split}
				f_i(\overline{x}) - f_i(\overline{y}) &= f_i\left( x_1, x_2, \ldots, x_n \right) - f_i\left( x_1, x_2, \ldots, y_n \right) + f_i\left( x_1, x_2, \ldots, y_n \right) - \\
				& \qquad f_i\left( x_1, x_2, \ldots, y_{n-1}, y_n \right) + f_i\left( x_1, x_2, \ldots, y_{n-1}, y_n \right) - \cdots - \\
				& \qquad f_i\left( x_1, \ldots, x_{j-1}, y_j, y_{j+1}, \ldots, y_n \right) + f_i\left( x_1, \ldots, x_{j-1}, y_j, y_{j+1}, \ldots, y_n \right) - \cdots - \\
				& \qquad f_i\left( x_1, y_2, \ldots, y_n \right) + f_i\left( x_1, y_2, \ldots, y_n \right) - f_i\left( y_1, y_2, \ldots, y_n \right) \\
				%
				&= g_n(x_n) - g_n(y_n) + g_{n-1}(x_{n-1}) - g_{n-1}(y_{n-1}) + g_{n-2}(x_{n-1}) - \cdots - \\
				& \qquad g_j(y_j) + g_{j-1}(x_{j-1}) - \cdots - g_2(y_2) + g_1(x_1) - g_1(y_1)
			\end{split}
		\end{equation*}

		Ya que $ f_i $ es diferenciable en $ U $, se tiene que, para todo $ j = 1, 2, \ldots, n $, $ g_j $ también lo es. Así, por el Teorema del Valor Medio, para cada $ j = 1, 2, \ldots, n $ existe $ a_j \in \bigl[ \min \left\lbrace x_j, y_j \right\rbrace, \max \left\lbrace x_j, y_j \right\rbrace \bigr] $ tal que \mbox{$ g_j(x_j) - g_j(y_j) = g_j(a_j) \left( x_j - y_j \right) $}. De este modo,

		\begin{equation*}
			\begin{split}
				f_i(\overline{x}) - f_i(\overline{y}) &= g'_n(a_n) \left( x_n - y_n \right) + g'_{n-1}(a_{n-1}) \left( x_{n-1} - y_{n-1} \right) + \cdots + g'_j(a_j) \left( x_j - y_j \right) + \cdots + \\
				& \qquad g'_1(a_1) \left( x_1 - y_1 \right) \\
				%
				&= \dfrac{\partial f_i}{\partial x_1} (\overline{z_1}) \left( x_1 - y_1 \right) + \dfrac{\partial f_i}{\partial x_2} (\overline{z_2}) \left( x_2 - y_2 \right) + \cdots + \dfrac{\partial f_i}{\partial x_j} (\overline{z_j}) \left( x_j - y_j \right) + \cdots + \\
				& \qquad \dfrac{\partial f_i}{\partial x_n} (\overline{z_n}) \left( x_n - y_n \right)
			\end{split}
		\end{equation*}

		donde $ \overline{z_j} = (x_1, \ldots, x_{j-1}, a_j, y_{j+1}, \ldots, y_n) $. Posteriormente,

		\begin{equation*}
			\begin{split}
				f_i(\overline{x}) - f_i(\overline{y}) - \sum_{j=1}^{n} \dfrac{\partial f_i}{\partial x_j} (\overline{x}) \left( x_j - y_j \right) &= \dfrac{\partial f_i}{\partial x_1} (\overline{z_1}) \left( x_1 - y_1 \right) - \dfrac{\partial f_i}{\partial x_1} (\overline{x}) \left( x_1 - y_1 \right) + \dfrac{\partial f_i}{\partial x_2} (\overline{z_2}) \left( x_2 - y_2 \right) - \\
				& \qquad \dfrac{\partial f_i}{\partial x_2} (\overline{x}) \left( x_2 - y_2 \right) + \cdots + \dfrac{\partial f_i}{\partial x_j} (\overline{z_j}) \left( x_j - y_j \right) - \\
				& \qquad \dfrac{\partial f_i}{\partial x_j} (\overline{x}) \left( x_j - y_j \right) + \cdots + \dfrac{\partial f_i}{\partial x_n} (\overline{z_n}) \left( x_n - y_n \right) - \\
				& \qquad \dfrac{\partial f_i}{\partial x_n} (\overline{x}) \left( x_n - y_n \right)
			\end{split}
		\end{equation*}

		\begin{equation*}
			\begin{split}
				\phantom{f_i(\overline{x}) - f_i(\overline{y}) - \sum_{j=1}^{n} \dfrac{\partial f_i}{\partial x_j} (\overline{x}) \left( x_j - y_j \right)} &= \left( \dfrac{\partial f_i}{\partial x_1} (\overline{z_1}) - \dfrac{\partial f_i}{\partial x_1} (\overline{x}) \right) \left( x_1 - y_1 \right) + \\
				& \qquad \left( \dfrac{\partial f_i}{\partial x_2} (\overline{z_2}) - \dfrac{\partial f_i}{\partial x_2} (\overline{x}) \right) \left( x_2 - y_2 \right) + \cdots + \\
				& \qquad \left( \dfrac{\partial f_i}{\partial x_j} (\overline{z_j}) - \dfrac{\partial f_i}{\partial x_j} (\overline{x}) \right) \left( x_j - y_j \right) + \cdots + \\
				& \qquad \left( \dfrac{\partial f_i}{\partial x_n} (\overline{z_n}) - \dfrac{\partial f_i}{\partial x_n} (\overline{x}) \right) \left( x_n - y_n \right)
			\end{split}
		\end{equation*}

		\begin{equation*}
			\begin{split}
				\Longrightarrow \left\lvert f_i(\overline{x}) - f_i(\overline{y}) - \sum_{j=1}^{n} \dfrac{\partial f_i}{\partial x_j} (\overline{x}) \left( x_j - y_j \right) \right\rvert &= \left\lvert \left( \dfrac{\partial f_i}{\partial x_1} (\overline{z_1}) - \dfrac{\partial f_i}{\partial x_1} (\overline{x}) \right) \left( x_1 - y_1 \right) + \right. \\
				& \qquad \left( \dfrac{\partial f_i}{\partial x_2} (\overline{z_2}) - \dfrac{\partial f_i}{\partial x_2} (\overline{x}) \right) \left( x_2 - y_2 \right) + \cdots + \\
				& \qquad \left( \dfrac{\partial f_i}{\partial x_j} (\overline{z_j}) - \dfrac{\partial f_i}{\partial x_j} (\overline{x}) \right) \left( x_j - y_j \right) + \cdots + \\
				& \qquad \left. \left( \dfrac{\partial f_i}{\partial x_n} (\overline{z_n}) - \dfrac{\partial f_i}{\partial x_n} (\overline{x}) \right) \left( x_n - y_n \right) \right\rvert \\
				%
				&\leq \left\lvert \dfrac{\partial f_i}{\partial x_1} (\overline{z_1}) - \dfrac{\partial f_i}{\partial x_1} (\overline{x}) \right\rvert \left\lvert x_1 - y_1 \right\rvert + \\
				& \qquad \left\lvert \dfrac{\partial f_i}{\partial x_2} (\overline{z_2}) - \dfrac{\partial f_i}{\partial x_2} (\overline{x}) \right\rvert \left\lvert x_2 - y_2 \right\rvert + \cdots + \\
				& \qquad \left\lvert \dfrac{\partial f_i}{\partial x_j} (\overline{z_j}) - \dfrac{\partial f_i}{\partial x_j} (\overline{x}) \right\rvert \left\lvert x_j - y_j \right\rvert + \cdots + \\
				& \qquad \left\lvert \dfrac{\partial f_i}{\partial x_n} (\overline{z_n}) - \dfrac{\partial f_i}{\partial x_n} (\overline{x}) \right\rvert \left\lvert x_n - y_n \right\rvert \\
				%
				&\leq \left\lvert \dfrac{\partial f_i}{\partial x_1} (\overline{z_1}) - \dfrac{\partial f_i}{\partial x_1} (\overline{x}) \right\rvert \left\lVert \overline{x} - \overline{y} \right\rVert + \\
				& \qquad \left\lvert \dfrac{\partial f_i}{\partial x_2} (\overline{z_2}) - \dfrac{\partial f_i}{\partial x_2} (\overline{x}) \right\rvert \left\lVert \overline{x} - \overline{y} \right\rVert + \cdots + \\
				& \qquad \left\lvert \dfrac{\partial f_i}{\partial x_j} (\overline{z_j}) - \dfrac{\partial f_i}{\partial x_j} (\overline{x}) \right\rvert \left\lVert \overline{x} - \overline{y} \right\rVert + \cdots + \\
				& \qquad \left\lvert \dfrac{\partial f_i}{\partial x_n} (\overline{z_n}) - \dfrac{\partial f_i}{\partial x_n} (\overline{x}) \right\rvert \left\lVert \overline{x} - \overline{y} \right\rVert \\
				%
				&= \Biggl( \left\lvert \dfrac{\partial f_i}{\partial x_1} (\overline{x}) - \dfrac{\partial f_i}{\partial x_1} (\overline{z_1}) \right\rvert + \left\lvert \dfrac{\partial f_i}{\partial x_2} (\overline{x}) - \dfrac{\partial f_i}{\partial x_2} (\overline{z_2}) \right\rvert + \cdots + \Biggr. \\
				& \qquad \Biggl. \left\lvert \dfrac{\partial f_i}{\partial x_j} (\overline{x}) - \dfrac{\partial f_i}{\partial x_j} (\overline{z_j}) \right\rvert + \cdots + \left\lvert \dfrac{\partial f_i}{\partial x_n} (\overline{x}) - \dfrac{\partial f_i}{\partial x_n} (\overline{z_n}) \right\rvert \Biggr) \left\lVert \overline{x} - \overline{y} \right\rVert
			\end{split}
		\end{equation*}

		\begin{equation*}
			\begin{split}
				\Longrightarrow \dfrac{\left\lvert f_i(\overline{x}) - f_i(\overline{y}) - \displaystyle \sum_{j=1}^{n} \dfrac{\partial f_i}{\partial x_j}(\overline{x}) \left( x_j - y_j \right) \right\rvert}{\left\lVert \overline{x} - \overline{y} \right\rVert} &\leq \left\lvert \dfrac{\partial f_i}{\partial x_1} (\overline{x}) - \dfrac{\partial f_i}{\partial x_1} (\overline{z_1}) \right\rvert + \left\lvert \dfrac{\partial f_i}{\partial x_2} (\overline{x}) - \dfrac{\partial f_i}{\partial x_2} (\overline{z_2}) \right\rvert + \cdots + \\
				& \quad \; \left\lvert \dfrac{\partial f_i}{\partial x_j} (\overline{x}) - \dfrac{\partial f_i}{\partial x_j} (\overline{z_j}) \right\rvert + \cdots + \left\lvert \dfrac{\partial f_i}{\partial x_n} (\overline{x}) - \dfrac{\partial f_i}{\partial x_n} (\overline{z_n}) \right\rvert
			\end{split}
		\end{equation*}

		Después, como las derivadas parciales son continuas en $ U $ se da que $ \displaystyle \lim_{\overline{x} \to \overline{y}} \left\lvert \dfrac{\partial f_i}{\partial x_j} (\overline{x}) - \dfrac{\partial f_i}{\partial x_j} (\overline{y}) \right\rvert = 0 $, y puesto que $ \overline{z_j} = (x_1, \ldots, x_{j-1}, a_j, y_{j+1}, \ldots, y_n) $ con $ a_j \in \bigl[ \min \left\lbrace x_j, y_j \right\rbrace, \max \left\lbrace x_j, y_j \right\rbrace \bigr] $, se tiene que $ \overline{z_j} $ tiende a $ \overline{y} $ conforme $ \overline{x} $ tiende a $ \overline{y} $, para todo $ j = 1, 2, \ldots, n $. De esta manera, 

		\begin{equation*}
			\begin{split}
				0 \leq \lim_{\overline{x} \to \overline{y}} \dfrac{ \phantom{|} \left\lvert f_i(\overline{x}) - f_i(\overline{y}) - \displaystyle \sum_{j=1}^{n} \dfrac{\partial f_i}{\partial x_j} (\overline{x}) \left( x_j - y_j \right) \right\rvert \phantom{|}}{ \left\lVert \overline{x} - \overline{y} \right\rVert } &\leq \lim_{\overline{x} \to \overline{y}} \Biggl( \left\lvert \dfrac{\partial f_i}{\partial x_1} (\overline{x}) - \dfrac{\partial f_i}{\partial x_1} (\overline{z_1}) \right\rvert + \Biggr. \left\lvert \dfrac{\partial f_i}{\partial x_2} (\overline{x}) - \right. \\
				& \left. \qquad \dfrac{\partial f_i}{\partial x_2} (\overline{z_2}) \right\rvert + \cdots + \left\lvert \dfrac{\partial f_i}{\partial x_j} (\overline{x}) - \dfrac{\partial f_i}{\partial x_j} (\overline{z_j}) \right\rvert + \cdots + \\
				& \qquad \Biggl. \left\lvert \dfrac{\partial f_i}{\partial x_n} (\overline{x}) - \dfrac{\partial f_i}{\partial x_n} (\overline{z_n}) \right\rvert \Biggr) \\
				%
				&= \lim_{\overline{x} \to \overline{y}} \left\lvert \dfrac{\partial f_i}{\partial x_1} (\overline{x}) - \dfrac{\partial f_i}{\partial x_1} (\overline{z_1}) \right\rvert + \lim_{\overline{x} \to \overline{y}} \left\lvert \dfrac{\partial f_i}{\partial x_2} (\overline{x}) - \right. \\
				& \qquad \left. \dfrac{\partial f_i}{\partial x_2} (\overline{z_2}) \right\rvert + \cdots + \lim_{\overline{x} \to \overline{y}} \left\lvert \dfrac{\partial f_i}{\partial x_j} (\overline{x}) - \dfrac{\partial f_i}{\partial x_j} (\overline{z_j}) \right\rvert + \cdots + \\
				& \qquad \lim_{\overline{x} \to \overline{y}} \left\lvert \dfrac{\partial f_i}{\partial x_n} (\overline{x}) - \dfrac{\partial f_i}{\partial x_n} (\overline{z_n}) \right\rvert \\
				%
				&= \lim_{\overline{x} \to \overline{y}} \left\lvert \dfrac{\partial f_i}{\partial x_1} (\overline{x}) - \dfrac{\partial f_i}{\partial x_1} (\overline{y}) \right\rvert + \lim_{\overline{x} \to \overline{y}} \left\lvert \dfrac{\partial f_i}{\partial x_2} (\overline{x}) - \right. \\
				& \qquad \left. \dfrac{\partial f_i}{\partial x_2} (\overline{y}) \right\rvert + \cdots + \lim_{\overline{x} \to \overline{y}} \left\lvert \dfrac{\partial f_i}{\partial x_j} (\overline{x}) - \dfrac{\partial f_i}{\partial x_j} (\overline{y}) \right\rvert + \cdots + \\
				& \qquad \lim_{\overline{x} \to \overline{y}} \left\lvert \dfrac{\partial f_i}{\partial x_n} (\overline{x}) - \dfrac{\partial f_i}{\partial x_n} (\overline{y}) \right\rvert \\
				&= 0
			\end{split}
		\end{equation*}

		De esta manera, $ \displaystyle \lim_{\overline{x} \to \overline{y}} \dfrac{ \phantom{|} \left\lvert f_i(\overline{x}) - f_i(\overline{y}) - \displaystyle \sum_{j=1}^{n} \dfrac{\partial f_i}{\partial x_j} (\overline{x}) \left( x_j - y_j \right) \right\rvert \phantom{|}}{ \left\lVert \overline{x} - \overline{y} \right\rVert } = 0 \quad \forall \, i = 1, \ldots, m $.

		Por último, 

		\begin{equation*}
			0 \leq \lim_{\overline{x} \to \overline{y}} \dfrac{ \left\lVert f(\overline{x}) - f(\overline{y}) - Df(\overline{y}) \left( \overline{x} - \overline{y} \right) \right\rVert }{ \left\lVert \overline{x} - \overline{y} \right\rVert } \leq \sum_{i=1}^{m} \lim_{\overline{x} \to \overline{y}} \dfrac{ \phantom{|} \left\lvert f_i(\overline{x}) - f_i(\overline{y}) - \displaystyle \sum_{j=1}^{n} \dfrac{\partial f_i}{\partial x_j} (\overline{x}) \left( x_j - y_j \right) \right\rvert \phantom{|}}{ \left\lVert \overline{x} - \overline{y} \right\rVert } = 0
		\end{equation*}

		Ya que $ \overline{y} $ fue arbitrario, se concluye que $ f $ es diferenciable en $ U $. \hfill $ \blacksquare $
	\end{ejercicio}

%--------------------------------------------------------------------------------------------------------------------

	\begin{ejercicio}{conjuntos}
		Sean $ A, B \subseteq \real $ no vacíos. Si $ a \leq b $ para todo $ a \in A $ y para todo $ b \in B $, entonces $ A $ está acotada superiormente y $ B $ está acotado inferiormente, y además, $ \sup (A) \leq \inf (B) $.

		\tcblower
		
		\textbf{Demostración.} 

		Sea $ b \in B $, ya que $ a \leq b $ para todo $ a \in A $, se tiene que $ A $ está acotado superiormente. De igual forma, sea $ a \in A $ como $ a \leq b $ para todo $ b \in B $, se da que $ B $ está acotado inferiormente. Además, dado que $ A $ y $ B $ son no vacíos, se obtiene que el supremo y el ínfimo de $ A $ y $ B $ existen, respectivamente. Como todo elemento de $ B $ es cota superior de $ A $ se tiene que $ \sup (A) $ es una cota inferior de $ B $. Por lo tanto, $ \sup (A) \leq \inf (B) $, pues $ \inf(B) $ es la mayor cota inferior de $ B $.
	\end{ejercicio}
	
%--------------------------------------------------------------------------------------------------------------------

	\begin{ejercicio}{suma}
		Si $ f \in \rsi{\alpha} $ en $ [a,b] $ y $ a < c < b $ entonces $ f \in \rsi{\alpha} $ en $ [a,c] $ y en $ [c,b] $ y

		\begin{equation*}
			\int_{a}^{b} f \; \mathrm{d} \alpha = \int_{a}^{c} f \; \mathrm{d} \alpha + \int_{c}^{b} f \; \mathrm{d} \alpha
		\end{equation*}

		\tcblower

		\textbf{Demostración.} 

		\textbf{Afirmación.} $ f \in \rsi{\alpha} $ en $ [a,c] $ y en $ [c,b] $.

		Sea $ \varepsilon > 0 $. Ya que $ f \in \rsi{\alpha} $ en $ [a,b] $ existe $ P_\varepsilon = \left\lbrace a = x_0, x_1, \ldots, x_n = b \right\rbrace \in $ {\large $ \gamma_{_{[a,b]}} $} tal que \\ $ U(f,P_\varepsilon,\alpha) - L(f,P_\varepsilon,\alpha) < \varepsilon $. Luego, para cada $ j = 1, \ldots, n $ sean $ M_j = \sup \left\lbrace f(x) \talque x \in \left[ x_{j-1}, x_j \right] \right\rbrace $ y $ m_j = \inf \left\lbrace f(x) \talque x \in \left[ x_{j-1}, x_j \right] \right\rbrace $. Ya que $ a < c < b $, existe $ i = 1, \ldots, n $ tal que $ x_{i-1} < c \leq x_i $, por lo que se definen $ M'_c = \sup \left\lbrace f(x) \talque x \in \left[ x_{i-1}, c \right] \right\rbrace $, $ M''_c = \sup \left\lbrace f(x) \talque x \in \left[ c, x_i \right] \right\rbrace $, $ m'_c = \inf \left\lbrace f(x) \talque x \in \left[ x_{i-1}, c \right] \right\rbrace $ y $ m''_c = \inf \left\lbrace f(x) \talque x \in \left[ c, x_i \right] \right\rbrace $.

		Después, considerando las particiones $ P'_\varepsilon = \left\lbrace a = x_0, x_1, \ldots, x_{i-1}, c \right\rbrace \in $ {\large $ \gamma_{_{[a,c]}} $} y \\ $ P''_\varepsilon = \left\lbrace c, x_i, x_{i+1}, \ldots, x_n \right\rbrace \in $ {\large $ \gamma_{_{[c,b]}} $}, se tiene que
		
		\begin{align*}
			&\mathcal{U} \left( f, P'_\varepsilon, \alpha \right) = \sum_{j=1}^{i-1} M_j \Delta \alpha_j + M'_c \left[ \alpha(c) - \alpha\left( x_{i-1} \right) \right], \\
			%
			&\mathcal{U} \left( f, P''_\varepsilon, \alpha \right) = M''_c \left[ \alpha(x_i) - \alpha\left( c \right) \right] + \sum_{j=i+1}^{n} M_j \Delta \alpha_j, \\
			%
			&\mathcal{L} \left( f, P'_\varepsilon, \alpha \right) = \sum_{j=1}^{i-1} m_j \Delta \alpha_j + m'_c \left[ \alpha(c) - \alpha\left( x_{i-1} \right) \right] \qquad \mbox{y} \\
		    %
			&\mathcal{L} \left( f, P''_\varepsilon, \alpha \right) = m''_c \left[ \alpha(x_i) - \alpha\left( c \right) \right] + \sum_{j=i+1}^{n} m_j \Delta \alpha_j.
		\end{align*}

		Puesto que $ \left[ x_{i-1}, c \right] \cup \left[ c, x_i \right] = \left[ x_{i-1}, x_i \right] $ se da que $ M'_c \leq M_i, M''_c \leq M_i, m_i \leq m'_c $ y $ m_i \leq m''_c $. Así,

		\begin{align*}
			\mathcal{U} \left( f, P'_\varepsilon, \alpha \right) + \mathcal{U} \left( f, P''_\varepsilon, \alpha \right) &= \sum_{j=1}^{i-1} M_j \Delta \alpha_j + M'_c \left[ \alpha(c) - \alpha\left( x_{i-1} \right) \right] + M''_c \left[ \alpha(x_i) - \alpha\left( c \right) \right] + \\
			& \qquad \sum_{j=i+1}^{n} M_j \Delta \alpha_j \\
			%
			&\leq \sum_{j=1}^{i-1} M_j \Delta \alpha_j + M_i \left[ \alpha(c) - \alpha\left( x_{i-1} \right) \right] + M_i \left[ \alpha(x_i) - \alpha\left( c \right) \right] + \\
			& \qquad \sum_{j=i+1}^{n} M_j \Delta \alpha_j \\
			%
			&= \sum_{j=1}^{n} M_j \Delta \alpha_j \\
			%
			&= U(f,P_\varepsilon,\alpha)
		\end{align*}
		
		y

		\begin{align*}
			- \mathcal{L} \left( f, P'_\varepsilon, \alpha \right) - \mathcal{L} \left( f, P''_\varepsilon, \alpha \right) &= - \sum_{j=1}^{i-1} m_j \Delta \alpha_j - m'_c \left[ \alpha(c) - \alpha\left( x_{i-1} \right) \right] - m''_c \left[ \alpha(x_i) - \alpha\left( c \right) \right]  \\
			& \qquad - \sum_{j=i+1}^{n} m_j \Delta \alpha_j \\
			%
			&\leq - \sum_{j=1}^{i-1} m_j \Delta \alpha_j - m_i \left[ \alpha(c) - \alpha\left( x_{i-1} \right) \right] - m_i \left[ \alpha(x_i) - \alpha\left( c \right) \right] - \\
			& \qquad\sum_{j=i+1}^{n} m_j \Delta \alpha_j \\
			%
			&= \sum_{j=1}^{n} m_j \Delta \alpha_j \\
			%
			&= -\mathcal{L} (f,P_\varepsilon,\alpha)
		\end{align*}

		De esta manera, 
		
		\begin{align*}
			&\mathcal{U} \left( f, P'_\varepsilon, \alpha \right) + \mathcal{U} \left( f, P''_\varepsilon, \alpha \right) - \mathcal{L} \left( f, P'_\varepsilon, \alpha \right) - \mathcal{L} \left( f, P''_\varepsilon, \alpha \right) \leq U(f,P_\varepsilon,\alpha) -\mathcal{L} (f,P_\varepsilon,\alpha) < \varepsilon \\
			%
			&\Longrightarrow \mathcal{U} \left( f, P'_\varepsilon, \alpha \right) - \mathcal{L} \left( f, P'_\varepsilon, \alpha \right) + \mathcal{U} \left( f, P''_\varepsilon, \alpha \right) - \mathcal{L} \left( f, P''_\varepsilon, \alpha \right) < \varepsilon \\
			%
			&\Longrightarrow \mathcal{U} \left( f, P'_\varepsilon, \alpha \right) - \mathcal{L} \left( f, P'_\varepsilon, \alpha \right) < \varepsilon \quad \mbox{ y } \quad \mathcal{U} \left( f, P''_\varepsilon, \alpha \right) - \mathcal{L} \left( f, P''_\varepsilon, \alpha \right) < \varepsilon
		\end{align*}
		
		Por lo que $ f \in \rsi{\alpha} $ en $ [a,c] $ y en $ [c,b] $. 

		Ahora, sea $ P = \left\lbrace a = x_0, x_1, \ldots, x_n = b \right\rbrace \in $ {\large $ \gamma_{_{[a,b]}} $}, procediendo como antes, se da que existen $ P' = \left\lbrace a = x_0, x_1, \ldots, x_{i-1}, c \right\rbrace \in $ {\large $ \gamma_{_{[a,c]}} $} y $ P'' = \left\lbrace c, x_i, x_{i+1}, \ldots, x_n = b \right\rbrace \in $ {\large $ \gamma_{_{[c,b]}} $} tales que $ \mathcal{U} \left( f, P', \alpha \right) + \mathcal{U} \left( f, P'', \alpha \right) \leq \mathcal{U} \left( f, P, \alpha \right) $ y $ \mathcal{L} \left( f, P', \alpha \right) + \mathcal{L} \left( f, P'', \alpha \right) \geq \mathcal{L} \left( f, P, \alpha \right) $. Posteriormente, se tiene que

		\begin{align*}
			&\mathcal{L} \left( f, P', \alpha \right) \leq \int_{a}^{c} f \; \mathrm{d} \alpha \leq \mathcal{U} \left( f, P', \alpha \right) \quad \hbox{y} \quad \mathcal{L} \left( f, P'', \alpha \right) \leq \int_{c}^{b} f \; \mathrm{d} \alpha \leq \mathcal{U} \left( f, P'', \alpha \right) \\
			%
			&\Longrightarrow \mathcal{L} \left( f, P', \alpha \right) + \mathcal{L} \left( f, P'', \alpha \right) \leq \int_{a}^{c} f \; \mathrm{d} \alpha + \int_{c}^{b} f \; \mathrm{d} \alpha \leq \mathcal{U} \left( f, P', \alpha \right) + \mathcal{U} \left( f, P'', \alpha \right) \\
			%
			&\Longrightarrow \mathcal{L} \left( f, P, \alpha \right) \leq \int_{a}^{c} f \; \mathrm{d} \alpha + \int_{c}^{b} f \; \mathrm{d} \alpha \leq \mathcal{U} \left( f, P, \alpha \right)
		\end{align*}

		Como $ P $ fue arbitraria, se obtiene que $ \displaystyle \int_{a}^{b} f \; \mathrm{d} \alpha = \int_{a}^{c} f \; \mathrm{d} \alpha + \int_{c}^{b} f \; \mathrm{d} \alpha $.
	\end{ejercicio}

%--------------------------------------------------------------------------------------------------------------------

	\begin{ejercicio}{convergenciaycontinuidad}
		Sea $ f \colon (X, \mathrm{d}_X) \to (Y, \mathrm{d}_Y) $. Si $ f $ es continua y $ \lbrace p_n \rbrace_{n \in \nat} $ es una sucesión en $ X $ tal que $ p_n \to p $, entonces $ f(p_n) \to f(p) $.

		\tcblower

		\textbf{Demostración.}

		Sea $ \varepsilon > 0 $. Como $ f $ es continua, existe $ \delta > 0 $ tal que si $ x \in X $ y $ \mathrm{d}_X(x,p) < \delta $ entonces \\ $ \mathrm{d}_Y \bigl( f(x),f(p) \bigr) < \varepsilon $. Luego, ya que $ p_n \to p $, para $ \delta $ existe $ N \in \nat $ tal que $ \mathrm{d}_X(p_n, p) < \delta $, para todo $ n \geq N $. Así, $ \mathrm{d}_Y \bigl( f(p_n), f(p) \bigr) < \varepsilon $, para todo $ n \geq N $. 

		Por lo tanto, $ f(p_n) \to f(p) $.
	\end{ejercicio}

%--------------------------------------------------------------------------------------------------------------------

	\begin{definicion}{funcioncontractiva}
		Sea $ f \colon (X, \mathrm{d}_X) \to (Y, \mathrm{d}_Y) $ una función entre espacios métricos. Se dice que $ f $ es \textbf{contractiva} si existe $ c \in [0,1) $ tal que 

		\begin{equation*}
			\mathrm{d}_Y \bigl( f(x), f(y) \bigr) \leq c \mathrm{d}_X(x,y) \quad \paratodo x, y \in X
		\end{equation*}
	\end{definicion}

%--------------------------------------------------------------------------------------------------------------------

	\begin{ejercicio}{contractivaesuniformementecontinua}
		Si $ f \colon (X, \mathrm{d}_X) \to (Y, \mathrm{d}_Y) $ es contractiva entonces $ f $ es uniformemente continua.

		\tcblower

		\textbf{Demostración.}

		Sea $ \varepsilon > 0 $. Ya que $ f $ es contractiva, para todo $ x, y \in X $ tal que $ \mathrm{d}_X(x, y) < \varepsilon $, existe $ c \in [0,1) $ tal que $ \mathrm{d}_Y \bigl( f(x), f(y) \bigr) \leq c \mathrm{d}_X(x,y) < c \varepsilon < \varepsilon $. 

		Por lo tanto, $ f $ es uniformemente continua.
	\end{ejercicio}

%--------------------------------------------------------------------------------------------------------------------

	\begin{definicion}{puntofijo}
		Sea $ f \colon A \to B $ y $ x \in A $. Se dice que $ x $ es un \textbf{punto fijo} si $ f(x) = x $.
	\end{definicion}

%--------------------------------------------------------------------------------------------------------------------

	\begin{ejercicio}{contractivaentoncesfijo}
		Sea $ X $ un espacio completo. Si $ f \colon (X, \mathrm{d}) \to (X, \mathrm{d}) $ es contractiva entonces $ f $ tiene un único punto fijo.

		\tcblower

		\textbf{Demostración.}

		Primero se demostrará que $ f $ tiene punto fijo. Sea $ x_0 \in X $, se define la sucesión $ \lbrace x_n \rbrace_{n \in \nat \cup \{0\}} $ como \\ $ x_{n+1} = f(x_n) $. 

		\textbf{Afirmación 1.} Para cada $ n \in \nat $ existe $ c \in [0,1) $ tal que $ \mathrm{d} (x_{n-1}, x_n) \leq c^n \mathrm{d} (x_0, x_1) $. 

		\hfill \begin{minipage}{0.98\linewidth}
			\setlength{\parskip}{1mm}

			Procediendo por inducción:

			Para $ n = 1 $ se da que $ \mathrm{d}(x_0, x_1) \leq c \mathrm{d}(x_0, x_1) $, para cualquier $ c \in [0,1) $.

			Luego, suponiendo que para $ n = k $ existe $ c_0 \in [0,1) $ tal que $ \mathrm{d} (x_{k-1}, x_k) \leq c_0^k \mathrm{d} (x_0, x_1) $. Como $ f $ es contractiva, para $ n = k + 1 $ se tiene que

			\begin{align*}
				\mathrm{d} (x_k, x_{k+1}) &= \mathrm{d} \bigl( f(x_{k-1}), f(x_k) \bigr) & \\
				%
				&\leq c_1 \mathrm{d} (x_{k-1}, x_k) & \bigl( \mbox{para algún $ c_1 \in [0,1) $} \bigr) \\
				%
				&\leq c_1 c_0^k \mathrm{d} (x_0, x_1) & \bigl( \mbox{por hipótesis de inducción} \bigr) \\
				%
				&\leq c^{k+1} \mathrm{d} (x_0, x_1) & \bigl( \mbox{donde $ c = \max \{ c_1, c_0 \} $}\bigr)
			\end{align*}

			Por lo tanto, para cada $ n \in \nat $ existe $ c \in [0,1) $ tal que $ \mathrm{d} (x_{n-1}, x_n) \leq c^n \mathrm{d} (x_0, x_1) $.
		\end{minipage}

		\textbf{Afirmación 2.} Para cada $ n, m \in \nat \cup \lbrace 0 \rbrace $ tal que $ n < m $, existe $ c \in [0,1) $ tal que \\ $ \mathrm{d} (x_n, x_m) \leq \dfrac{\mathrm{d} (x_0, x_1)}{1-c} c^{n+1} $. 

		\hfill \begin{minipage}{0.98\linewidth}
			\setlength{\parskip}{1mm}

			Sean $ n, m \in \nat \cup \lbrace 0 \rbrace $ tales que $ n < m $, se tiene que

			\begin{equation*}
				\mathrm{d} (x_n, x_m) \leq \sum_{i=n+1}^{m} \mathrm{d} (x_{i-1}, x_i) \\
			\end{equation*}

			Por la afirmación anterior, para cada $ i = n+1, n+2, \ldots, m $ existe $ c_i \in [0,1) $ tal que 
			
			\begin{align*}
				\mathrm{d} (x_n, x_m) &\leq \sum_{i=n+1}^{m} c_i^i \mathrm{d} (x_0, x_1) & \\
				%
				&\leq \sum_{i=n+1}^{m} c^i \mathrm{d} (x_0, x_1) & \mbox{donde $ c = \max \{ c_{n+1}, c_{n+2}, \ldots, c_m \} $} \\
				%
				&= \mathrm{d} (x_0, x_1) \dfrac{c^{n+1}-c^{m+1}}{1-c} & \\
				%
				&\leq \dfrac{\mathrm{d} (x_0, x_1)}{1-c} c^{n+1} &
			\end{align*}

			Por lo tanto, para cada $ n, m \in \nat \cup \lbrace 0 \rbrace $ tal que $ n < m $, existe $ c \in [0,1) $ tal que \\ $ \mathrm{d} (x_n, x_m) \leq \dfrac{\mathrm{d} (x_0, x_1)}{1-c} c^{n+1} $. 
		\end{minipage}

		\textbf{Afirmación 3.} Para cada $ \varepsilon > 0 $ existe $ N \in \nat $ tal que $ \dfrac{\mathrm{d} (x_0, x_1)}{1-c} c^N < \varepsilon $ para cualquier $ c \in [0,1) $. 

		\hfill \begin{minipage}{0.98\linewidth}
			\setlength{\parskip}{1mm}

			Sea $ \varepsilon > 0 $ y $ c \in [0,1) $ para $ \varepsilon \dfrac{1-c}{\mathrm{d} (x_0, x_1)} > 0 $, existe un $ N \in \nat $ tal que $ c^N < \varepsilon \dfrac{1-c}{\mathrm{d} (x_0, x_1)} $. Por lo tanto, $ \dfrac{\mathrm{d} (x_0, x_1)}{1-c} c^N < \varepsilon $. 
		\end{minipage}

		Ahora, sea $ \varepsilon > 0 $, por la afirmación anterior, existe $ N \in \nat $ tal que $ \dfrac{\mathrm{d} (x_0, x_1)}{1-c} c^N < \varepsilon $ para todo $ c \in [0,1) $. Luego, para todo $ m > n \geq N $, y por la Afirmación 2, existe $ c \in [0,1) $ tal que \vspace{-3mm}

		\begin{equation*}
			\mathrm{d} (x_n, x_m) \leq \dfrac{\mathrm{d} (x_0, x_1)}{1-c} c^{n+1} \leq \dfrac{\mathrm{d} (x_0, x_1)}{1-c} c^{N+1} < \dfrac{\mathrm{d} (x_0, x_1)}{1-c} c^N < \varepsilon
		\end{equation*}
		
		De este modo, $ \lbrace x_n \rbrace_{n \in \nat \cup \{0\}} $ es de Cauchy y como $ X $ es un espacio completo $ x_n \to x $, para algún $ x \in X $. Luego, por el Ejercicio \ref{ejer:contractivaesuniformementecontinua}, $ f $ es uniformemente continua, lo cual implica que es continua. Así, $ f(x_n) \to f(x) $, pero por cómo se definió $ \lbrace x_n \rbrace_{n \in \nat \cup \{0\}} $, se tiene que $ \lbrace x_n \rbrace_{n \in \nat} = \lbrace f(x_n) \rbrace_{n \in \nat \cup \{0\}} $, por lo que $ f(x_n) \to x $. De esta forma, $ f(x) = x $, de modo que $ f $ tiene un punto fijo.

		Posteriormente, suponiendo que $ f $ tiene dos puntos fijos distintos, $ x $ y $ y $, se da que \vspace{-3mm}
		
		\begin{equation*}
			\mathrm{d} (x, y) = \mathrm{d} \bigl( f(x), f(y) \bigr) \leq c \mathrm{d} (x, y) \qquad \mbox{para algún } c \in [0,1)
		\end{equation*}

		y como $ \mathrm{d} (x, y) > 0 $, se da que $ 1 \leq c $, lo cual no puede ser. Por lo tanto, $ f $ tiene un único punto fijo.
	\end{ejercicio}
	
%--------------------------------------------------------------------------------------------------------------------

	\begin{ejercicio}{cantorintegrable}
		Sean $ \mathcal{C} $ el conjunto de Cantor, $ f \colon [0,1] \to \real $ acotada y continua en $ [0,1] \setminus \mathcal{C} $. Demostrar que $ f \in \mathcal{R} $ en $ [0,1] $.

		\tcblower

		\textbf{Demostración.}

		Sean $ \epsilon > 0 $, $ M = \sup \bigl\lbrace \lvert f(x) \rvert \talque x \in [0,1] \bigr\rbrace $ y $ \displaystyle U = \bigcup_{i=1}^m (a_i, b_i) $ una cubierta abierta finita de $ \mathcal{C} $ tal que $ \displaystyle \sum_{i=1}^{m} (b_i - a_i) < \dfrac{\varepsilon}{4M} $. Ya que $ [0,1] \setminus \mathcal{C} $ es cerrado y acotado, se obtiene que $ [0,1] \setminus \mathcal{C} $ es cerrado, por lo cual $ f $ es uniformemente continua en $ [0,1] \setminus \mathcal{C} $. Así, para $ \dfrac{\varepsilon}{2} > 0 $, existe $ \delta > 0 $ tal que si $ x, y \in [0,1] \setminus \mathcal{C} $ y $ \lvert x - y \rvert < \delta $ entonces $ \lvert f(x) - f(y) \rvert < \dfrac{\varepsilon}{2} $.

		Luego, sea $ P = \lbrace 0 = t_0, t_1, \ldots, t_n = 1 \rbrace $ una partición de $ [0,1] $ tal que $ t_j - t_{j-1} < \delta $ y $ a_j, b_j \in P $ para todo $ j = 1, \ldots, n $ y sea $ A = \left\lbrace j \in \lbrace 1, \ldots, n \rbrace \talque [t_{j-1}, t_j] \subseteq [0,1] \setminus U \right\rbrace $ se tiene que 

		\begin{align*}
			\mathcal{U} (f, P) - \mathcal{L} (f, P) &= \sum_{j=1}^{n} (M_j - m_j)(t_j - t_{j-1}) \\
			%
			&= \sum_{j \in A} (M_j - m_j)(t_j - t_{j-1}) + \sum_{j \notin A} (M_j - m_j)(t_j - t_{j-1}) \\
			%
			&\leq \dfrac{\varepsilon}{2} \sum_{j \in A} (t_j - t_{j-1}) + 2M \sum_{j \notin A} (t_j - t_{j-1}) \\
			%
			&< \dfrac{\varepsilon}{2} + 2M \dfrac{\varepsilon}{4M} \\
			%
			&= \dfrac{\varepsilon}{2} + \dfrac{\varepsilon}{2} \\
			%
			&= \varepsilon
		\end{align*}

		Por lo tanto, $ f \in \mathcal{R} $ en $ [0,1] $.
	\end{ejercicio}

%--------------------------------------------------------------------------------------------------------------------

	\begin{ejercicio}{impropiacoincideintegral}
		Sea $f$ una función real definida sobre $ [0,1] $ tal que $ f \in \mathcal{R} $ sobre $ [c,1] $ para cada $ c > 0 $. Se define $ \displaystyle \intg{0}{1}{f(x)}{x} = \lim_{c \to 0} \intg{c}{1}{f(x)}{x} $ si el límite existe y es finito.

		Si $ f \in [0,1] $ mostrar que esta definición de la integral coincide con la definición antigua.
		
		\tcblower

		\textbf{Demostración.}

		Sean $ \varepsilon > 0 $, $ M = \sup \bigl\lbrace |f(x)| \talque x \in [0,1] \bigr\rbrace $ y $ c \in (0,1) \cap \left( 0, \dfrac{\varepsilon}{6M} \right) $ fijo. Como $ f \in \mathcal{R} $ en $ [0,1] $ existe $ P_\varepsilon = \left\lbrace 0 = t_0, t_1, \ldots, t_n = 1 \right\rbrace $ una partición de $ [0,1] $ tal que $ \mathcal{U} (f,P_\epsilon) - \mathcal{L} (f,P_\epsilon) < \dfrac{\varepsilon}{3} $. Supongamos, sin pérdida de generalidad, que $ c \in P_\varepsilon $, es decir, que $ c = t_j $ para algún $ j = 1, 2, \ldots, n-1 $.

		Luego, sea $ Q = \left\lbrace c = t_j, t_{j+1}, \ldots, t_n = 1 \right\rbrace $, $Q$ es partición de $ [0,1] $ y se tiene que:

		\begin{align*}
			\bigl\lvert \mathcal{U} (f,Q) - \mathcal{L} (f,Q) \bigr\rvert &= \left\lvert \sum_{i=j+1}^{n} (M_i - m_i) (t_i - t_{i-1}) \right\rvert & \\
			%
			&\leq \sum_{i=j+1}^{n} \bigl( \left\lvert M_i \right\rvert + \left\lvert m_i \right\rvert \bigr) (t_i - t_{i-1}) & \\
			%
			&\leq \sum_{i=j+1}^{n} 2M (t_i - t_{i-1}) & \\
			%
			&< 2M & \\
			%
			&< 2 \dfrac{\varepsilon}{6c} & \left( \mbox{pues } c < \dfrac{\varepsilon}{6M} \Longrightarrow M < \dfrac{\varepsilon}{6c} \right) \\
			%
			&= \dfrac{\varepsilon}{3c} & \\
			%
			&< \dfrac{\varepsilon}{3} & \left( \mbox{pues } 0 < c < 1 \Longrightarrow 0 < \dfrac{1}{c} < 1 \Longrightarrow 0 < \dfrac{1}{3c} < \dfrac{1}{3} \right)
		\end{align*}

		Después,

		\begin{align*}
			\left\lvert \mathcal{L} (f,Q) - \mathcal{U} (f,P_\varepsilon) \right\rvert &= \left\lvert \sum_{i=j+1}^{n} m_i (t_i - t_{i-1}) - \sum_{i=1}^{n} M_i (t_i - t_{i-1}) \right\rvert \\
			%
			&\leq \sum_{i=j+1}^{n} \left\lvert m_i \right\rvert (t_i - t_{i-1}) + \sum_{i=1}^{n} \left\lvert M_i \right\rvert (t_i - t_{i-1}) \\
			%
			&\leq \sum_{i=j+1}^{n} M (t_i - t_{i-1}) + \sum_{i=1}^{n} M (t_i - t_{i-1}) \\
			%
			&< 2M \\
			%
			&< \dfrac{\varepsilon}{3}
		\end{align*}

		Así,

		\begin{align*}
			\left\lvert \intg{c}{1}{f(x)}{x} - \intg{0}{1}{f(x)}{x} \right\rvert &\leq \left\lvert \intg{c}{1}{f(x)}{x} - \mathcal{L} (f,Q) \right\rvert + \bigl\lvert \mathcal{L} (f,Q) - \mathcal{U} (f,P_\varepsilon) \bigr\rvert + \\
			&\quad \left\lvert \mathcal{U} (f,P_\varepsilon) - \intg{0}{1}{f(x)}{x} \right\rvert \\
			%
			&\leq \left\lvert \mathcal{U} (f,Q) - \mathcal{L} (f,Q) \right\rvert + \left\lvert \mathcal{L} (f,Q) - \mathcal{U} (f,P_\varepsilon) \right\rvert + \left\lvert \mathcal{U} (f,P_\varepsilon) - \mathcal{L} (f,P_\varepsilon) \right\rvert \\
			%
			&< \dfrac{\varepsilon}{3} + \dfrac{\varepsilon}{3} + \dfrac{\varepsilon}{3} \\
			%
			&= \varepsilon
		\end{align*}

		Por lo tanto, $ \displaystyle \lim_{c \to 0} \intg{c}{1}{f(x)}{x} = \intg{0}{1}{f(x)}{x} $.
	\end{ejercicio}

\end{document}