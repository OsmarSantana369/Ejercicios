\documentclass[fleqn, 12pt]{article}
\usepackage[a4paper, margin = 21mm]{geometry}
\usepackage[spanish]{babel}
\usepackage{parskip}
\usepackage{amsmath, amsfonts, amssymb}
\usepackage{enumerate}
\usepackage{graphicx}

\usepackage[proportional,scaled=1.064]{erewhon}
\usepackage[erewhon,vvarbb,bigdelims]{newtxmath}
\usepackage[T1]{fontenc}
\renewcommand*\oldstylenums[1]{\textosf{#1}}

\expandafter\def\expandafter\normalsize\expandafter{%
    \setlength\abovedisplayskip{-12pt}%
    \setlength\belowdisplayskip{4pt}%
}

\begin{document}
	\begin{enumerate}
		\setcounter{enumi}{2}
		\item Demuestre que el hiperplano $ H = \left\{ x \colon px = k \right\} $ y un semiespacio $ H^+ = \left\{ x \colon px \geq k \right\} $ son conjuntos convexos.
		
		\textbf{Demostración.}

		Sean $ x,y \in H $ y $ \lambda \in \left[ 0,1 \right] $, se tiene que

		\begin{align*}
			& px = k \quad \text{y} \quad py = k \\
			& \Longrightarrow \lambda px = \lambda k \quad \text{y} \quad \left( 1 - \lambda \right) py = \left( 1 - \lambda \right) k \\
			& \Longrightarrow \lambda px + \left( 1 - \lambda \right) py = \lambda k + \left( 1 - \lambda \right) k \\
			& \Longrightarrow p \left( \lambda x + \left( 1 - \lambda \right) y \right) = k \\
			& \Longrightarrow \lambda x + \left( 1 - \lambda \right) y \in H
		\end{align*}

		Por lo tanto, $ H $ es convexo.

		Ahora, sean $ a,b \in H^+ $ y $ \lambda \in \left[ 0,1 \right] $, se obtiene que

		\begin{align*}
			& pa \geq k \quad \text{y} \quad pb \geq k \\
			& \Longrightarrow \lambda pa \geq \lambda k \quad \text{y} \quad \left( 1 - \lambda \right) pb \geq \left( 1 - \lambda \right) k \\
			& \Longrightarrow \lambda pa + \left( 1 - \lambda \right) pb \geq \lambda k + \left( 1 - \lambda \right) k \\
			& \Longrightarrow p \left( \lambda a + \left( 1 - \lambda \right) b \right) \geq k \\
			& \Longrightarrow \lambda a + \left( 1 - \lambda \right) b \in H^+
		\end{align*}

		Por lo tanto, $ H^+ $ es convexo.

		\item Considere el conjunto $ X = \left\{ \left( x_1, x_2 \right) \colon x_1 \geq 0, x_2 \geq 0, x_1 + x_2 \geq 2, x_2 \leq 4 \right\} $. Encuentre un hiperplano $ H $ tal que $ X $ y el punto $ \left( 3, -2 \right) $ estén en lados diferentes del hiperplano. Escriba la ecuación del hiperplano.
		
		\textbf{Solución.}

		Sea $ H = \left\{ (x,y) \in \mathbb{R}^2 \colon \begin{pmatrix} 0 & -1 \end{pmatrix} \begin{pmatrix} x \\ y \end{pmatrix} = k \right\} $
		\item 
	\end{enumerate}
\end{document}