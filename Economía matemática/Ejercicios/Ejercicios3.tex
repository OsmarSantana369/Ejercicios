\documentclass[12pt]{article}
\usepackage[utf8]{inputenc}
\usepackage[spanish]{babel}
\usepackage{amsmath, amssymb}
\usepackage{geometry}
\geometry{a4paper, margin=1.5cm}

\begin{document}
\begin{enumerate}
    \item Suponiendo que las siguientes funciones son lineales, proporcione una interpretación económica de la pendiente de la función:
    
    \begin{enumerate}
        \item $F(q)$ representa los ingresos por la producción de $q$ unidades de producto.
        \item $G(x)$ representa el costo de adquirir $x$ unidades de un bien.
		\item $H(p)$ representa la cantidad del bien consumida cuando su precio es $p$
        \item $C(Y)$ representa el consumo nacional total cuando el ingreso nacional es $Y$.
        \item $S(Y)$ representa el ahorro nacional total cuando el ingreso nacional es $Y$.
    \end{enumerate}

    \item El costo total de un fabricante es $C(x) = 0.1x^3 + 0.25x^2 + 300x + 100$ dólares, donde $x$ es el nivel de producción. Estime el efecto sobre el costo total de un aumento len el nivel de producción de 1 unidad.

    \item Obten una fórmula para calcular el tiempo que tarda el dinero en triplicarse en una cuenta bancaria que paga intereses a una tasa $ r $ capitalizada continuamente. ¿Cuánto tardarán \$500 en convertirse en \$600 si la tasa de interés es del 5\% capitalizada continuamente?
    
	\item Con una tasa de interés del 10\% anual, ¿cuál de las siguientes opciones tiene el mayor valor presente: \$215 dentro de 2 años, \$100 después de cada uno de los dos próximos dos años o \$100 ahora y \$95 dentro de dos años?

    \item Suponiendo una tasa de interés del 10\% capitalizable continuamente, ¿cuál es el valor presente de una anualidad que paga \$500 al año; durante los próximos 5 años; indefinidamente?
	
	\begin{align*}
		PV = \dfrac{A(1 - e^{-rN})}{e^r - 1}
	\end{align*}
	
	Para calcular el valor actual de una anualidad que paga \$A dólares al año indefinidamente

    \begin{align*}
        \text{Si} N \to \infty \quad S_{N \to \infty} \sim PV = \frac{A}{e^r - 1}
    \end{align*}

    \item Supongamos que posees un libro cuyo valor dentro de $ t $ años será de $ B(t) = 2^{\sqrt{t}} $ dólares. Suponiendo una tasa de interés constante del 5\%, ¿cuándo es el mejor momento para vender el libro e invertir las ganancias?

    \item Un comerciante de vinos posee una caja que puede vender por $ V(t) = Ke^{\sqrt{t}} $ dólares dentro de $t$ años. Si no hay costos de almacenamiento y la tasa de interés es $ r $, ¿Cuánto debería, el comerciante vender el vino?

    \item El valor de un terreno comprado con fines especulativosaumenta según la fórmula $V(t)$. Si la tasa de interés es del 10\%, ¿cuánto tiempo se debe mantener el terreno para maximizar su valor presente?
    


	

    \item Las ganancias anuales de una empresa aumentaron un 20\% entre 2010 y 2011, pero luego disminuyeron un 17\% entre 2011 y 2012. ¿Cuál fue la ganancia promedio anual?

    \item Demanda de educación en EE.UU. en 1990 fue 15192 millones. Estime para 1992.

    \item Si la demanda y oferta se expresan por:
    \[
        D = 150 - 12p, \quad S = 20 + 2p
    \]
    Halle el precio de equilibrio $p^*$ y la cantidad de equilibrio $Q^*$.

    \item Suponga que se aplica un impuesto de \$2 por unidad al productor. ¿Cómo cambia el equilibrio?

    \item Halle el beneficio marginal y la interpretación de:
    \[
        S'(x) = 100 + 0.01x^2, \quad P'(x) = 0.17x
    \]

    \item Sea la ecuación:
    \[
        Q^2 = 3^t
    \]
    ¿Cómo se interpreta cuando $t = 10$?

    \item Sea $C(Q) = 2.4Q^2 - Q$, ¿qué cantidad maximiza los beneficios?

    \item Sea $Q(t) = 1000t$, ¿es sensible aumentar la producción?

    \item Sea la ecuación para IS (ahorro de inversión):
    \[
        Y = \frac{A}{g} \left( \frac{1}{T_s} - \frac{1}{T_d} \right)
    \]
    Donde $L$, $D$ es la propensión marginal, $g$ la sensibilidad, $T_s$ y $T_d$ tasas de interés.

    \item Escriba el sistema IS-LM en forma matricial:
    \[
        Y = \frac{M_0 + I}{i}
    \]
\end{enumerate}

\section*{Elasticidad}
\begin{enumerate}
    \setcounter{enumi}{21}
    \item ¿Cuándo se dice que una demanda es elástica, inelástica o unitaria?

    \item Si el precio de un artículo cambia de \$4 a \$6 y la cantidad demandada disminuye de 30 a 20, calcule la elasticidad.

    \item Si la demanda de pantalones aumenta 10\% cuando el precio disminuye 10\%, ¿es elástica?

    \item Si la demanda de pantalones aumenta 22\% cuando el precio disminuye 10\%, ¿es elástica?

    \item Calcule la elasticidad cruzada entre el precio de la naranja y la cantidad demandada de jugo.

    \item Calcule la elasticidad de la oferta cuando el precio aumenta de \$125 a \$135 por par.
\end{enumerate}

\textbf{Tabla:}
\[
\begin{array}{|c|c|}
\hline
\text{Precio por par} & \text{Cantidad ofrecida} \\
\hline
125 & 20 \\
130 & 22 \\
135 & 24 \\
\hline
\end{array}
\]

\end{document}