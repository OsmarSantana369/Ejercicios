\documentclass[12pt]{article}
\usepackage[spanish]{babel}
\usepackage[margin = 21mm, top = 24mm]{geometry}
\usepackage{amsmath, amssymb, amsfonts}
\usepackage{parskip}
\usepackage{tcolorbox}       % Agregar cajas con colores
\usepackage{colortbl}
\usepackage{array, tabularx}       % Insertar tablas en cajas de texto
\usepackage{enumerate}
\usepackage{multirow}
\usepackage{multicol}
\usepackage{graphicx}

\usepackage[proportional,scaled=1]{erewhon}
\usepackage[erewhon,vvarbb,bigdelims]{newtxmath}
\usepackage[T1]{fontenc}
\renewcommand*\oldstylenums[1]{\textosf{#1}}

\title{}
\author{Osmar Dominique Santana Reyes}
\date{\today}

\expandafter\def\expandafter\normalsize\expandafter{%
    \setlength\abovedisplayskip{-9pt}%
    \setlength\belowdisplayskip{6pt}%
}

\newcommand{\paratodo}{\, \forall \,}
\newcommand{\existe}{\exists \,}
\newcommand{\talque}{\; \middle| \;}
\newcommand{\nat}{\mathbb{N}}
\newcommand{\ent}{\mathbb{Z}}

\begin{document}
	\hfill \today

	\textbf{Teoría de números}

	Osmar Dominique Santana Reyes

	\begin{enumerate}
		\item Sea $ m = 1, 2, 4, p^\alpha, 2p^\alpha $, con $ p $ primo impar y $ \alpha \in \nat $, y $ a \in \ent $. $ x^n \equiv a (\bmod m) $ tiene $ (n, \phi(m)) $ soluciones si y solo si $ a^{\frac{\phi(m)}{(n, \phi(m))}} \equiv 1 (\bmod m) $.
		
		\textbf{Demostración.}

		Sea $ g $ raíz primitiva módulo $ m $ e $ i \in \lbrace 1, 2, \ldots, \phi(m) \right $ tal que $ g^i \equiv a (\bmod m) $.
		
		Primero, si se supone que $ x^n \equiv a (\bmod m) $ tiene $ (n, \phi(m)) $ soluciones, entonces sea $ x_0 $ una de estas soluciones y $ u $ tal que $ x_0 \equiv g^u (\bmod m) $, se tiene que

		\begin{equation*}
			g^i \equiv a \equiv x_0^n \equiv g^{un} (\bmod m)
		\end{equation*}

		\item 
	\end{enumerate}
\end{document}