\documentclass[fleqn]{article}
\usepackage[spanish]{babel}
\usepackage{amsmath, amssymb, amsfonts}
\usepackage{parskip}

\begin{document}
    {\Huge \textbf{Regiones elementales y sus fronteras}}

    \textbf{Region elemental (en \double{R}^3):} Es aquella en la que una de las variables esta acotada, inferior y superiormente, por funciones, $ \gamma_1 $ y $ \gamma_2 $ (donde \gamma_2 \geq \gamma_1), que dependen de las otras dos variables y su dominio es una región elemental en $ \double{R}^2 $. 

    Ejemplos:

    \begin{enumerate}
    
    \item $ D = \lbrace (x,y,z) \in \double{R}^3 \middle | \gamma_1(x,y) \leq z \leq \gamma_2(x,y)

    \end{enumerate}
\end{document}