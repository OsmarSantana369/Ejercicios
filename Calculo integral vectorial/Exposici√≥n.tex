\documentclass[fleqn]{article}
\usepackage[spanish]{babel}
\usepackage{amsmath, amssymb, amsfonts}
\usepackage{parskip}

\begin{document}
    {\Huge \textbf{Regiones elementales y sus fronteras}}

    \textbf{Region elemental (en $ \double{R}^3 $):} Es aquella en la que una de las variables esta acotada, inferior y superiormente, por funciones, $ \gamma_1 $ y $ \gamma_2 $ (donde $ \gamma_2 \geq \gamma_1 $), que dependen de las otras dos variables y su dominio es una región elemental en $ \double{R}^2 $. 

    Ejemplos:

    \begin{enumerate}

        \item $ S = \lbrace (x,y,z) \in \double{R}^3 \middle | 0 \leq x \leq \sqrt{2}, 0 \leq y \leq \sqrt{2 - x^2} \text{ y } x^2 + y^2 \leq z \leq 2 \rbrace $

    \end{enumerate}

    \textbf{Superficie cerrada:} Sea $ S $ una región elemental en $ \double{R}^3 $. A $ \partial S $ se le llama superficie cerrada. Las superficies $ S_1, S_2, \dots , S_6 $ que conforman a $ \partial S $ y pueden representarse como gráficas de funciones de $ \double{R}^2 $ a $ \double{R} $ son sus caras.

    Por ejemplo, considerando la región $ S $ del ejemplo anterior, sus caras son 


    Definición: Sea $ S $ una superficie cerrada. Si el vector normal a $ S $ apunta al exterior de este, entonces la superficie tiene orientación exterior. 


    Si el vector normal a $ S $ apunta al interior de este, entonces la superficie tiene orientación interior.

    Sea $ F $ el campo de velocidades de un fluido
\end{document}