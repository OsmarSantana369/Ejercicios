\documentclass[fleqn]{article}
\usepackage[utf8]{inputenc}
\usepackage[spanish, es-lcroman]{babel}
\usepackage[margin = 15mm]{geometry}
\usepackage{parskip}
\usepackage{amsmath, amssymb, amsfonts}
\usepackage{enumerate}

\usepackage[proportional,scaled=1]{erewhon}
\usepackage[erewhon,vvarbb,bigdelims]{newtxmath}
\usepackage[T1]{fontenc}
\renewcommand*\oldstylenums[1]{\textosf{#1}}

\expandafter\def\expandafter\normalsize\expandafter{%
    \setlength\abovedisplayskip{-9pt}%
    \setlength\belowdisplayskip{5pt}%
}

\newcommand{\despar}[3]{\dfrac{\mathrm{d}^#1 #2}{\mathrm{d} #3^#1}}
\newcommand{\depar}[2]{\dfrac{\mathrm{d} #1}{\mathrm{d} #2}}
\newcommand{\real}{\mathbb{R}}
\newcommand{\nat}{\mathbb{N}}

\begin{document}
	\bfseries
	Ecuaciones Diferenciables Parciales \\
	Tarea 5 \\
	Osmar Dominique Santana Reyes \\
	No. de cuenta: 2125197 \\

	\begin{enumerate}[I.]
		\item Verifique si el problema dado es un problema de auto valor de S-L regular.
		
		\begin{enumerate}
			\item $ f''(x) + \lambda f(x) = 0 $, $ 0 < x < 1 $, $ f(0) + 2 f'(0) = 0 $, $ f'(1) = 0 $.
			
			Solución.
			
			\normalfont

			Sean $ p(x) = 1, q(x) = 0, \sigma(x) = 1, \alpha_1 = \beta_2 = 1, \beta_1 = 2 $ y $ \alpha_2 = 0 $, la ED se puede escribir como $ \left( p(x) f'(x) \right)' + \left[ q(x) + \lambda \sigma(x) \right] f(x) = 0 $, con $ 0 < x < 1, \; \alpha_1 f(0) + \beta_1 f'(0) = 0 $, $ \alpha_2 f(1) + \beta_2 f'(1) = 0 $. Además,

			\begin{enumerate}[i)]
				\item $ p(x) = 1, p'(x) = 0, q(x) = 0 $ y $ \sigma(x) = 1 $ son funciones reales y continuas sobre $ 0 < x < 1 $,
				\item $ p(x) > 0 $ y $ \sigma(x) > 0 $ para todo $ 0 < x < 1 $ y
				\item $ \alpha_1, \alpha_2, \beta_1, \beta_2 \in \real $ son tales que $ \alpha_1^2 + \beta_1^2 = 5 \neq 0 $ y $ \alpha_2^2 + \beta_2^2 = 1 \neq 0 $
			\end{enumerate}

			Por lo tanto, este es un problema de auto valor de S-L regular.

			\bfseries

			%------------------------------------------------------------------------------------------------------------

			\item $ f''(x) - x f(x) + \lambda (x^2 + 1) f(x) = 0 $, $ 0 < x < 1 $, $ f(0) = 0 $, $ f'(1) = 0 $.
			
			Solución.
			
			\normalfont

			Sean $ p(x) = 1, q(x) = -x, \sigma(x) = x^2 + 1, \alpha_1 = \beta_2 = 1 $ y $ \beta_1 = \alpha_2 = 0 $, la ED se puede escribir como $ \left( p(x) f'(x) \right)' + \left[ q(x) + \lambda \sigma(x) \right] f(x) = 0 $, con $ 0 < x < 1, \; \alpha_1 f(0) + \beta_1 f'(0) = 0 $, $ \alpha_2 f(1) + \beta_2 f'(1) = 0 $. Además,

			\begin{enumerate}[i)]
				\item $ p(x) = 1, p'(x) = 0, q(x) = -x $ y $ \sigma(x) = x^2 + 1 $ son funciones reales y continuas sobre $ 0 < x < 1 $,
				\item $ p(x) > 0 $ y $ \sigma(x) > 0 $ para todo $ 0 < x < 1 $ y
				\item $ \alpha_1, \alpha_2, \beta_1, \beta_2 \in \real $ son tales que $ \alpha_1^2 + \beta_1^2 = 1 \neq 0 $ y $ \alpha_2^2 + \beta_2^2 = 1 \neq 0 $
			\end{enumerate}

			Por lo tanto, este es un problema de auto valor de S-L regular.

		\end{enumerate}

		%----------------------------------------------------------------------------------------------------------------
		
		\bfseries
			
		\item Calcule los valores propios y las funciones propias del problema regular S-L dado.
		
		\begin{enumerate}
			\item $ f''(x) + \lambda f(x) = 0 $, $ 0 < x < \pi $, $ f(0) = 0 $, $ f'(\pi) = 0 $.
			
			Solución.
			
			\normalfont

			\begin{enumerate}[i)]
				\item Si $ \lambda < 0 $ entonces, sea $ \lambda = - \mu^2 $, con $ \mu \neq 0 $, se tiene que $ f''(x) - \mu^2 f(x) = 0 $
				tiene como solución general a $ f(x) = a \cosh(\mu x) + b \senh(\mu x) $. Luego, por las condiciones de frontera se da que
				
				\begin{equation*}
					0 = f(0) = a \cosh(\mu \cdot 0) + b \senh(\mu \cdot 0) = a
				\end{equation*}

				Así, $ f(x) = b \senh(\mu x) $ y 

				\begin{align*}
					0 = f'(\pi) = b \mu \cosh(\mu \pi) \Longrightarrow b = 0
				\end{align*}
				
				De esta forma, $ f(x) = 0 $.

				\item Si $ \lambda = 0 $, entonces la ED, queda como $ f''(x) = 0 $, por lo que $ f(x) = Ax + B $ y, por las condiciones de frontera, se obtiene que 
				
				\begin{align*}
					0 = f(0) = B \mbox{ y } 0 = f'(\pi) = A
				\end{align*}

				De este modo, $ f(x) = 0 $.
				
				\item Si $ \lambda > 0 $ entonces, sea $ \lambda = \mu^2 $, con $ \mu \neq 0 $, se tiene que $ f''(x) + \mu^2 f(x) = 0 $
				tiene como solución general a $ f(x) = a \cos(\mu x) + b \sen(\mu x) $. Luego, por las condiciones de frontera se da que
				
				\begin{equation*}
					0 = f(0) = a \cos(\mu \cdot 0) + b \sen(\mu \cdot 0) = a
				\end{equation*}

				Así, $ f(x) = b \sen(\mu x) $ y 

				\begin{align*}
					0 = f'(\pi) = b \mu \cos(\mu \pi) \, \Longrightarrow \, \cos(\mu \pi) = 0 \, \Longrightarrow \, \mu \pi = \dfrac{\pi}{2} + n \pi, \quad \mbox{con } n \in \nat
				\end{align*}
				
				Por lo tanto, los valores propios del problema son: $ \lambda_n = \dfrac{(2n + 1)^2}{4} $, \, con $ n \in \nat $. Mientras que las funciones propias son: $ f_n (x) = \sen \left( \dfrac{2n + 1}{2} x \right) $, \, con $ n \in \nat $.
			\end{enumerate}

			%------------------------------------------------------------------------------------------------------------

			\bfseries
			
			\item $ f''(x) + \lambda f(x) = 0 $, $ 0 < x < 1 $, $ f'(0) = 0 $, $ f(1) = 0 $.
			
			Solución.
			
			\normalfont

			\begin{enumerate}[i)]
				\item Si $ \lambda < 0 $ entonces, sea $ \lambda = - \mu^2 $, con $ \mu \neq 0 $, se tiene que $ f''(x) - \mu^2 f(x) = 0 $
				tiene como solución general a $ f(x) = a \cosh(\mu x) + b \senh(\mu x) $. Luego, por las condiciones de frontera se da que
				
				\begin{equation*}
					0 = f'(0) = a \mu \senh(\mu \cdot 0) + b \mu \cosh(\mu \cdot 0) = b \mu \, \Longrightarrow b = 0
				\end{equation*}

				Así, $ f(x) = a \cosh(\mu x) $ y 

				\begin{align*}
					0 = f(1) = a \cosh(\mu) \Longrightarrow a = 0
				\end{align*}
				
				De esta forma, $ f(x) = 0 $.

				\item Si $ \lambda = 0 $, entonces la ED, queda como $ f''(x) = 0 $, por lo que $ f(x) = Ax + B $ y, por las condiciones de frontera, se obtiene que 
				
				\begin{align*}
					0 = f'(0) = A \mbox{ y } 0 = f(1) = B
				\end{align*}

				De este modo, $ f(x) = 0 $.
				
				\item Si $ \lambda > 0 $ entonces, sea $ \lambda = \mu^2 $, con $ \mu \neq 0 $, se tiene que $ f''(x) + \mu^2 f(x) = 0 $
				tiene como solución general a $ f(x) = a \cos(\mu x) + b \sen(\mu x) $. Luego, por las condiciones de frontera se da que
				
				\begin{equation*}
					0 = f'(0) = -a \mu \sen (\mu \cdot 0) + b \mu \cos (\mu \cdot 0) = b \mu \Longrightarrow b = 0
				\end{equation*}

				Así, $ f(x) = a \cos(\mu x) $ y 

				\begin{align*}
					0 = f(1) = a \cos(\mu) \, \Longrightarrow \, \cos(\mu) = 0 \, \Longrightarrow \, \mu = \dfrac{\pi}{2} + n \pi, \quad \mbox{con } n \in \nat
				\end{align*}
				
				Por lo tanto, los valores propios del problema son: $ \lambda_n = \dfrac{\pi^2 (2n + 1)^2}{4} $, \, con $ n \in \nat $. Mientras que las funciones propias son: $ f_n (x) = \cos \left( \dfrac{2n + 1}{2} \pi x \right) $, \, con $ n \in \nat $.
			\end{enumerate}

			%------------------------------------------------------------------------------------------------------------

			\bfseries
			
			\item $ f''(x) + \lambda f(x) = 0 $, $ 0 < x < 1 $, $ f(0) - f'(0) = 0 $, $ f(1) = 0 $.
			
			Solución.
			
			\normalfont

			\begin{enumerate}[i)]
				\item Si $ \lambda < 0 $ entonces, sea $ \lambda = - \mu^2 $, con $ \mu \neq 0 $, se tiene que $ f''(x) - \mu^2 f(x) = 0 $
				tiene como solución general a $ f(x) = a \cosh(\mu x) + b \senh(\mu x) $. Luego, por las condiciones de frontera se da que
				
				\begin{equation*}
					0 = f(0) - f'(0) = a \cosh(\mu \cdot 0) + b \senh(\mu \cdot 0) - a \mu \senh(\mu \cdot 0) - b \mu \cosh(\mu \cdot 0) = a - b \mu \Longrightarrow a = b \mu
				\end{equation*}

				Así, $ f(x) = b \mu \cosh(\mu x) + b \senh(\mu x) $ y 

				\begin{align*}
					0 = f(1) = b \mu \cosh(\mu) + b \senh(\mu) \Longrightarrow b \left( \mu + \tanh(\mu) \right) = 0 \Longrightarrow b = 0
				\end{align*}
				
				De esta forma, $ f(x) = 0 $.

				\item Si $ \lambda = 0 $, entonces la ED, queda como $ f''(x) = 0 $, por lo que $ f(x) = Ax + B $ y, por las condiciones de frontera, se obtiene que 
				
				\begin{align*}
					0 = f(0) - f'(0) = B - A \mbox{ y } 0 = f(1) = A + B
				\end{align*}

				De este modo, $ A = B = 0 $ por lo que $ f(x) = 0 $.
				
				\item Si $ \lambda > 0 $ entonces, sea $ \lambda = \mu^2 $, con $ \mu \neq 0 $, se tiene que $ f''(x) + \mu^2 f(x) = 0 $
				tiene como solución general a $ f(x) = a \cos(\mu x) + b \sen(\mu x) $. Luego, por las condiciones de frontera se da que
				
				\begin{equation*}
					0 = f(0) - f'(0) = a \cos(\mu \cdot 0) + b \sen(\mu \cdot 0) + a \mu \sen (\mu \cdot 0) - b \mu \cos (\mu \cdot 0) = a - b \mu \Longrightarrow a = b \mu
				\end{equation*}

				Así, $ f(x) = b \mu \cos(\mu x) + b \sen(\mu x) $ y 

				\begin{align*}
					0 = f(1) = b \mu \cos(\mu) + b \sen(\mu) \, \Longrightarrow \, b \left( \mu + \tan(\mu) \right) = 0 \, \Longrightarrow \, \mu = - \tan (\mu)
				\end{align*}
				
				Por lo tanto, los valores propios del problema son todos los $ \mu_n^2 $ tales que $ \mu_n = - \tan (\mu_n) $, \, con $ n \in \nat $. Mientras que las funciones propias son: $ f_n (x) = \mu_n \cos(\mu_n x) + \sen(\mu_n x) $, \, con $ n \in \nat $.
			\end{enumerate}

			%------------------------------------------------------------------------------------------------------------

			\bfseries
			
			\item $ f''(x) + \lambda f(x) = 0 $, $ 0 < x < 1 $, $ f'(0) = 0 $, $ f(1) + f'(1) = 0 $.
			
			Solución.
			
			\normalfont

			\begin{enumerate}[i)]
				\item Si $ \lambda < 0 $ entonces, sea $ \lambda = - \mu^2 $, con $ \mu \neq 0 $, se tiene que $ f''(x) - \mu^2 f(x) = 0 $
				tiene como solución general a $ f(x) = a \cosh(\mu x) + b \senh(\mu x) $. Luego, por las condiciones de frontera se da que
				
				\begin{equation*}
					0 = f'(0) = a \mu \senh(\mu \cdot 0) + b \mu \cosh(\mu \cdot 0) = b \mu \Longrightarrow b = 0 
				\end{equation*}

				Así, $ f(x) = a \cosh(\mu x) $ y 

				\begin{align*}
					0 = f(1) + f'(1) = a \cosh(\mu) + a \mu \senh(\mu) \Longrightarrow a \left( 1 + \mu \tanh(\mu) \right) = 0 \Longrightarrow a = 0
				\end{align*}
				
				De esta forma, $ f(x) = 0 $.

				\item Si $ \lambda = 0 $, entonces la ED, queda como $ f''(x) = 0 $, por lo que $ f(x) = Ax + B $ y, por las condiciones de frontera, se obtiene que 
				
				\begin{align*}
					0 = f'(0) = A \mbox{ y } 0 = f(1) + f'(1) = B + 0 = B
				\end{align*}

				De este modo, $ A = B = 0 $ por lo que $ f(x) = 0 $.
				
				\item Si $ \lambda > 0 $ entonces, sea $ \lambda = \mu^2 $, con $ \mu \neq 0 $, se tiene que $ f''(x) + \mu^2 f(x) = 0 $
				tiene como solución general a $ f(x) = a \cos(\mu x) + b \sen(\mu x) $. Luego, por las condiciones de frontera se da que
				
				\begin{equation*}
					0 = f'(0) = -a \mu \sen(\mu \cdot 0) + b \mu \cos(\mu \cdot 0) = b \mu \Longrightarrow b = 0
				\end{equation*}

				Así, $ f(x) = a \cos(\mu x) $ y 

				\begin{align*}
					0 = f(1) + f'(1) = a \cos(\mu) - a \mu \sen(\mu) \, \Longrightarrow \, a \left( 1 - \mu \tan(\mu) \right) = 0 \, \Longrightarrow \, 1 = \mu \tan (\mu)
				\end{align*}
				
				Por lo tanto, los valores propios del problema son todos los $ \mu_n^2 $ tales que $ 1 = \mu_n \tan (\mu_n) $, \, con $ n \in \nat $. Mientras que las funciones propias son: $ f_n (x) = \cos(\mu_n x) $, \, con $ n \in \nat $.
			\end{enumerate}
			
		\end{enumerate}

		%----------------------------------------------------------------------------------------------------------------
		
		\bfseries
			
		\item Construya la expansión de la serie de Fourier generalizada para la función dada $U$ en las funciones propias.
		
		\begin{equation*}
			\left\lbrace \sen \left( \dfrac{(2n - 1) x}{2} \right) \right\rbrace_{n=1}^{\infty}
		\end{equation*}

		Del problema S-L con $ L = \pi $.
		
		\begin{enumerate}
			\item $ U(x) = 1 $, $ 0 \leq x \leq \pi $.
			
			Solución.
			
			\normalfont



			%------------------------------------------------------------------------------------------------------------

			\bfseries
			
			\item $ U(x) = 2x - 1 $, $ 0 \leq x \leq \pi $.
			
			Solución.
			
			\normalfont



		\end{enumerate}

		%----------------------------------------------------------------------------------------------------------------

		\bfseries
			
		\item Construya la expansión de la serie de Fourier generalizada para la función dada $U$ en las funciones propias.
		
		\begin{equation*}
			\left\lbrace \cos \left( \dfrac{(2n - 1) \pi x}{2} \right) \right\rbrace_{n=1}^{\infty}
		\end{equation*}

		Del problema S-L con $ L = 1 $.
		
		\begin{enumerate}
			\item $ U(x) = \dfrac{1}{2} $, $ 0 \leq x \leq 1 $.
			
			Solución.
			
			\normalfont



			%------------------------------------------------------------------------------------------------------------

			\bfseries
			
			\item $ U(x) = x + 1 $, $ 0 \leq x \leq 1 $.
			
			Solución.
			
			\normalfont



		\end{enumerate}
		
		%----------------------------------------------------------------------------------------------------------------

		\bfseries
			
		\item Usar el cociente de Rayleigh para obtener una cota superior (razonablemente precisa) del autovalor mínimo de los siguientes problemas.
		
		\begin{enumerate}
			\item $ \despar{2}{\phi}{x} + (\lambda - x^2) \phi = 0 $ con $ \depar{\phi}{x} (0) = 0 $ y $ \phi (1) = 0 $.
			
			Solución.
			
			\normalfont



			%------------------------------------------------------------------------------------------------------------

			\bfseries
			
			\item $ \despar{2}{\phi}{x} + (\lambda - x) \phi = 0 $ con $ \depar{\phi}{x} (0) = 0 $ y $ \depar{\phi}{x} (0) + 2 \phi (1) = 0 $.
			
			Solución.
			
			\normalfont



		\end{enumerate}
	\end{enumerate}
\end{document}