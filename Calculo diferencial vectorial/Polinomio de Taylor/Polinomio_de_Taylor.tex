\documentclass[fleqn, 12pt]{article}
\usepackage[a4paper, margin = 21mm]{geometry}
\usepackage[spanish]{babel}
\usepackage{parskip}
\usepackage{amsmath, amsfonts, amsthm}
\usepackage{enumerate}
\usepackage{graphicx}
\usepackage[p,osf]{scholax}
\usepackage[scaled=1.075,ncf,vvarbb]{newtxmath}

\newcommand{\derivadaparcial}[2]{\dfrac{\partial {#1}}{\partial {#2}}}
\newcommand{\derivadaparcialn}[3]{\dfrac{\partial^{#3} {#1}}{\partial {#2}^{#3}}}
\newcommand{\derivadaparcialnd}[3]{\dfrac{\partial^{2} {#1}}{\partial {#3} \partial {#2}}}

\begin{document}
    \begin{center}
        UNIVERSIDAD AUTÓNOMA DEL ESTADO DE MÉXICO \\
        FACULTAD DE CIENCIAS \\
        DEPARTAMENTO DE MATEMÁTICAS \\
        Cálculo Diferencial Vectorial \\
        Profesores: Dr. Félix Capulín Pérez \\
        Dr. Enrique Castañeda Alvarado \\
        Tarea: Polinomio de Taylor
    \end{center}

    Nombre: Osmar Dominique Santana Reyes \hfill No. de cuenta: $ 2125197 $

    En cada uno de los ejercicios, determinar la fórmula de Taylor de segundo orden para la función dada alrededor del punto $ x_0, y_0 $.

    \begin{enumerate}
%-------Ejercicio 1------------------------------------------------------------------------------------------------------
        \item $ f(x,y) = (x + y)^2 $, donde $ x_0 = 0 $, $ y_0 = 0 $.
        
        \textbf{Solución.}

        $ f(0,0) = (0 + 0)^2 = 0 $

        $ \derivadaparcial{f}{x} (0,0) = \left. 2(x + y) \right|_{(0,0)} = 0 $

        $ \derivadaparcial{f}{y} (0,0) = \left. 2(x + y) \right|_{(0,0)} = 0 $

        $ \derivadaparcialn{f}{x}{2} (0,0) = 2 $

        $ \derivadaparcialnd{f}{x}{y} (0,0) = 2 $

        $ \derivadaparcialn{f}{y}{2} (0,0) = 2 $

        Así, el Polinomio de Taylor es
        \begin{align*}
            f((x_0, y_0) + (h_1, h_2)) &= f(h_1, h_2) \\
            &= f(0,0) + h_1 \, \derivadaparcial{f}{x} (0,0) + h_2 \, \derivadaparcial{f}{y} (0,0) + \dfrac{1}{2} \left( h_1^2 \, \derivadaparcialn{f}{x}{2} (0,0) \right. + \\
            & \quad \left. h_1 h_2 \, \derivadaparcialnd{f}{x}{y} (0,0) + h_2 h_1 \, \derivadaparcialnd{f}{y}{x} (0,0) + h_2^2 \, \derivadaparcialn{f}{y}{2} (0,0) \right) \\
            &= \dfrac{1}{2} \left( 2 h_1^2 + 2 h_1 h_2 + 2 h_2 h_1 + 2 h_2^2 \right) \\
            &= h_1^2 + 2 h_1 h_2 + h_2^2
        \end{align*}
        Notemos que $ f(x,y) = (x + y)^2 = x^2 + 2xy + y^2 $. Por lo tanto, es igual a su Polinomio de Taylor.

%-------Ejercicio 2------------------------------------------------------------------------------------------------------
        \item $ f(x,y) = \dfrac{1}{x^2 + y^2 + 1} $, donde $ x_0 = 0 $, $ y_0 = 0 $.
        
        \textbf{Solución.}

        $ f(0,0) = \dfrac{1}{0^2 + 0^2 + 1} = 1 $

        $ \derivadaparcial{f}{x} (0,0) = \left. - \dfrac{2x}{(x^2 + y^2 + 1)^2} \right|_{(0,0)} = 0 $

        $ \derivadaparcial{f}{y} (0,0) = \left. - \dfrac{2y}{(x^2 + y^2 + 1)^2} \right|_{(0,0)} = 0 $
        \begin{align*}
            \hspace{-10mm} \derivadaparcialn{f}{x}{2} (0,0) &= \left. \dfrac{(x^2 + y^2 + 1)^2 (-2) + (2x)(2x)(2(x^2 + y^2 + 1))}{(x^2 + y^2 + 1)^4} \right|_{(0,0)} \\
            &= \left. \dfrac{-2(x^2 + y^2 + 1)^2 + 8x^2 (x^2 + y^2 + 1)}{(x^2 + y^2 + 1)^4} \right|_{(0,0)} \\
            &= \left. \dfrac{-2(x^2 + y^2 + 1) + 8x^2}{(x^2 + y^2 + 1)^3} \right|_{(0,0)} \\
            &= \left. \dfrac{6x^2 - 2y^2 - 2}{(x^2 + y^2 + 1)^3} \right|_{(0,0)} \\
            &= -2
        \end{align*}
        $ \derivadaparcialnd{f}{x}{y} (0,0) = \left. \dfrac{8xy(x^2 + y^2 + 1)}{(x^2 + y^2 + 1)^4} \right|_{(0,0)} = 0 $
        \begin{align*}
            \hspace{-10mm} \derivadaparcialn{f}{y}{2} (0,0) &= \left. - \dfrac{(x^2 + y^2 + 1)^2 (-2) + (2y)(2y)(2(x^2 + y^2 + 1))}{(x^2 + y^2 + 1)^4} \right|_{(0,0)} \\
            &= \left. - \dfrac{-2(x^2 + y^2 + 1)^2 + 8y^2 (x^2 + y^2 + 1)}{(x^2 + y^2 + 1)^4} \right|_{(0,0)} \\
            &= \left. - \dfrac{-2(x^2 + y^2 + 1) + 8y^2}{(x^2 + y^2 + 1)^3} \right|_{(0,0)} \\
            &= \left. - \dfrac{-2x^2 + 6y^2 - 2}{(x^2 + y^2 + 1)^3} \right|_{(0,0)} \\
            &= -2
        \end{align*}
        Así, el Polinomio de Taylor es
        \begin{align*}
            f((x_0, y_0) + (h_1, h_2)) &= f(h_1, h_2) \\
            &= f(0,0) + h_1 \, \derivadaparcial{f}{x} (0,0) + h_2 \, \derivadaparcial{f}{y} (0,0) + \dfrac{1}{2} \left( h_1^2 \, \derivadaparcialn{f}{x}{2} (0,0) \right. + \\
            & \quad \left. h_1 h_2 \, \derivadaparcialnd{f}{x}{y} (0,0) + h_2 h_1 \, \derivadaparcialnd{f}{y}{x} (0,0) + h_2^2 \, \derivadaparcialn{f}{y}{2} (0,0) \right) \\
            &= 1 + \dfrac{1}{2} \left( -2 h_1^2 - 2 h_2^2 \right) \\
            &= 1 - h_1^2 - h_2^2
        \end{align*}

%-------Ejercicio 3------------------------------------------------------------------------------------------------------
        \item $ f(x,y) = e^{x + y} $, donde $ x_0 = 0 $, $ y_0 = 0 $.
        
        \textbf{Solución.}

        $ f(0,0) = e^{0 + 0} = 1 $

        $ \derivadaparcial{f}{x} (0,0) = \left. e^{x + y} \right|_{(0,0)} = 1 $

        $ \derivadaparcial{f}{y} (0,0) = \left. e^{x + y} \right|_{(0,0)} = 1 $

        $ \derivadaparcialn{f}{x}{2} (0,0) = \left. e^{x + y} \right|_{(0,0)} = 1 $

        $ \derivadaparcialnd{f}{x}{y} (0,0) = \left. e^{x + y} \right|_{(0,0)} = 1 $

        $ \derivadaparcialn{f}{y}{2} (0,0) = \left. e^{x + y} \right|_{(0,0)} = 1 $

        Así, el Polinomio de Taylor es
        \begin{align*}
            f((x_0, y_0) + (h_1, h_2)) &= f(h_1, h_2) \\
            &= f(0,0) + h_1 \, \derivadaparcial{f}{x} (0,0) + h_2 \, \derivadaparcial{f}{y} (0,0) + \dfrac{1}{2} \left( h_1^2 \, \derivadaparcialn{f}{x}{2} (0,0) \right. + \\
            & \quad \left. h_1 h_2 \, \derivadaparcialnd{f}{x}{y} (0,0) + h_2 h_1 \, \derivadaparcialnd{f}{y}{x} (0,0) + h_2^2 \, \derivadaparcialn{f}{y}{2} (0,0) \right) \\
            &= 1 + h_1 + h_2 + \dfrac{1}{2} \left( h_1^2 + 2 h_1 h_2 + h_2^2 \right) \\
            &= 1 + h_1 + h_2 + h_1 h_2 + \dfrac{h_1^2 + h_2^2}{2}
        \end{align*}

%-------Ejercicio 4------------------------------------------------------------------------------------------------------
        \item $ f(x,y) = e^{-x^2 - y^2} \cos (xy) $, donde $ x_0 = 0 $, $ y_0 = 0 $.
        
        \textbf{Solución.}
        
        $ f(0,0) = e^{-0^2 - 0^2} \cos (0 \cdot 0) = 1 $

        $ \derivadaparcial{f}{x} (0,0) = \left. -2x e^{-x^2 - y^2} \cos (xy) + e^{-x^2 - y^2} \cos (xy) \right|_{(0,0)} = 1 $

        $ \derivadaparcial{f}{y} (0,0) = \left. e^{x + y} \right|_{(0,0)} = 1 $

        $ \derivadaparcialn{f}{x}{2} (0,0) = \left. e^{x + y} \right|_{(0,0)} = 1 $

        $ \derivadaparcialnd{f}{x}{y} (0,0) = \left. e^{x + y} \right|_{(0,0)} = 1 $

        $ \derivadaparcialn{f}{y}{2} (0,0) = \left. e^{x + y} \right|_{(0,0)} = 1 $

        Así, el Polinomio de Taylor es
        \begin{align*}
            f((x_0, y_0) + (h_1, h_2)) &= f(h_1, h_2) \\
            &= f(0,0) + h_1 \, \derivadaparcial{f}{x} (0,0) + h_2 \, \derivadaparcial{f}{y} (0,0) + \dfrac{1}{2} \left( h_1^2 \, \derivadaparcialn{f}{x}{2} (0,0) \right. + \\
            & \quad \left. h_1 h_2 \, \derivadaparcialnd{f}{x}{y} (0,0) + h_2 h_1 \, \derivadaparcialnd{f}{y}{x} (0,0) + h_2^2 \, \derivadaparcialn{f}{y}{2} (0,0) \right) \\
            &= 1 + h_1 + h_2 + \dfrac{1}{2} \left( h_1^2 + 2 h_1 h_2 + h_2^2 \right) \\
            &= 1 + h_1 + h_2 + h_1 h_2 + \dfrac{h_1^2 + h_2^2}{2}
        \end{align*}

%-------Ejercicio 5------------------------------------------------------------------------------------------------------
        \item $ f(x,y) = \sen (xy) + \cos (xy) $, donde $ x_0 = 0 $, $ y_0 = 0 $.
        
        \textbf{Solución.}


%-------Ejercicio 6------------------------------------------------------------------------------------------------------
        \item $ f(x,y) = e^{(x - 1)^2} \cos (y) $, donde $ x_0 = 1 $, $ y_0 = 0 $.
        
        \textbf{Solución.}


    \end{enumerate}
\end{document}