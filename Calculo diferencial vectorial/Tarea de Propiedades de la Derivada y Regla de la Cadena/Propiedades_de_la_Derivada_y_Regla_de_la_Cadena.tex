\documentclass[fleqn]{article}
\usepackage{amsmath, amssymb}
\usepackage{parskip}
\usepackage[a4paper, margin = 21mm]{geometry}
\usepackage{graphicx, subfig}

\begin{document}
    %Portada
    \begin{titlepage}
        
        \begin{figure*}[t]
            \subfloat{\includegraphics[width = 29mm]{UAEMex.png}}
            $\begin{array}{c}
                \textnormal{\LARGE \textbf{UNIVERSIDAD AUTÓNOMA DEL}} \\
                \textnormal{\LARGE \textbf{ESTADO DE MÉXICO}}
            \end{array}$
            \subfloat{\includegraphics[height = 29mm]{Fciencias.png}}
        \end{figure*}

        \centering
        {\LARGE \textbf{FACULTAD DE CIENCIAS}}

        \vspace{5mm}
        {\LARGE \textbf{LICENCIATURA EN MATEMÁTICAS}}

        \vspace{24mm}
        {\Large \textbf{APLICACIONES DEL CÁLCULO DIFERENCIAL VECTORIAL}}

        {\Large \textbf{CÁLCULO DIFERENCIAL VECTORIAL}}

        \vspace{7mm}
        {\Large \textbf{DR. ENRIQUE CASTAÑEDA ALVARADO}}

        {\Large \textbf{DR. FÉLIX CAPULÍN PÉREZ}}
        
        \vspace{24mm}
        {\Large \textbf{PROPIEDADES DE LA DERIVADA Y REGLA DE LA CADENA}}

        \vspace{30mm}
        {\Large \textbf{Osmar Dominique Santana Reyes}}

        \vspace{7mm}
        {\Large \textbf{Semestre: 2022B}}
        
        \vfill
        \raggedleft{\Large Fecha de Entrega: }
    \end{titlepage}
\end{document}