\documentclass[fleqn]{article}
\usepackage[a4paper, margin = 21mm]{geometry}
\usepackage{parskip}
\usepackage{amsmath, amssymb, amsfonts}

\begin{document}
    Osmar Dominique Santana Reyes

    Examen. Límite, continuidad y derivada.

    \begin{enumerate}
        %Primer ejercicio
        \item 
        \begin{enumerate}
            %Inciso a del primer ejercicio
            \item Afirmación: $ \lim_{(x,y) \to (0,0)} f(x,y) = 0 $.

            \textbf{Demostración.}

            Sean $ (x,y) \in \mathbb{R}^2 \setminus \left\lbrace \overline{0} \right\rbrace $, $ \epsilon > 0 $ y $ \delta = \epsilon $. Se tiene que

            $$ 0 \leq (\lvert x \rvert + \lvert y \rvert)^2 = \lvert x \rvert^2 - 2 \lvert xy \rvert + \lvert y \rvert^2 $$

            $$ \Longrightarrow 2 \lvert xy \rvert \leq \lvert x \rvert^2 + \lvert y \rvert^2 $$

            $$ \Longrightarrow \dfrac{\lvert xy \rvert}{x^2 + y^2} \leq \dfrac{1}{2} $$
            \begin{equation}
                \Longrightarrow \dfrac{\lvert x \rvert \cdot \lvert xy \rvert}{x^2 + y^2} \leq \dfrac{\lvert x \rvert}{2}
                \label{eq:1a1}
            \end{equation}
            
            Luego, si $ 0 < \lVert (x,y) - (0,0) \rVert < \delta $ entonces

            $$ \sqrt{x^2 + y^2} < \delta = \epsilon $$

            $$ \Longrightarrow \lvert x \rvert \leq \sqrt{x^2 + y^2} < \epsilon $$

            $$ \Longrightarrow \lvert \dfrac{\lvert x \rvert}{2} \rvert \leq \lvert x \rvert < \epsilon $$

            $$ \Longrightarrow \dfrac{\lvert x \rvert \cdot \lvert xy \rvert}{x^2 + y^2} \leq \lvert \dfrac{\lvert x \rvert}{2} \rvert < \epsilon $$

            $$ \Longrightarrow \dfrac{\lvert x^2 y \rvert}{x^2 + y^2} < \epsilon $$

            $$ \Longrightarrow \lvert \dfrac{x^2 y}{x^2 + y^2} - 0 \rvert < \epsilon $$

            $$ \therefore \lim_{(x,y) \to (0,0)} f(x,y) = 0 $$

            %Inciso b del primer ejercicio
            \item Afirmación: $ \lim_{(x,y) \to (0,0)} g(x,y) $ no existe.

            \textbf{Demostración.}

            Sean $ (x,y) \in \mathbb{R}^2 \setminus \left\lbrace \overline{0} \right\rbrace $, $ \epsilon > 0 $ y $ \delta = \sqrt{\epsilon} $. Si $ y = 0 $ y $ 0 < \lVert (x,y) - (0,0) \rVert < \delta $ entonces

            $$ \lVert (x,0) \rVert = \lvert x \rvert < \delta = \sqrt{\epsilon} $$

            $$ \lvert x \rvert < \sqrt{\epsilon} $$

            $$ \lvert x \rvert^2 < \epsilon $$

            $$ \lvert (x^2 + 1) - 1 \rvert < \epsilon $$

            $$ \therefore \lim_{(x,0) \to (0,0)} g(x,y) = 1 $$

            Ahora, sea $ \delta_1 = \sqrt{\epsilon - 1} $. Si $ x = y $ y $ 0 < \lVert (x,y) - (0,0) \rVert < \delta_1 $ entonces

            $$ \lVert (x,x) \rVert = \lvert x \rvert \sqrt{2} < \delta_1 = \sqrt{\epsilon - 1} $$

            $$ \Longrightarrow \lvert x \rvert < \lvert x \rvert \sqrt{2} < \sqrt{\epsilon - 1} $$

            $$ \Longrightarrow \lvert x \rvert < \sqrt{\epsilon - 1} $$

            $$ \Longrightarrow \lvert x \rvert^2 < \epsilon - 1 $$

            $$ \Longrightarrow \lvert x^2 \rvert + 1 < \epsilon $$

            $$ \Longrightarrow \lvert x^2 + 1 - 0 \rvert < \epsilon $$

            $$ \therefore \lim_{(x,x) \to (0,0)} g(x,y) = 0 $$

            Como los límites obtenidos difieren entonces $ \lim_{(x,y) \to (0,0)} g(x,y) $ no existe.
        \end{enumerate}

        %Segundo ejercicio
        \item \textbf{Demostración.}

        Sean $ (x,y) \in \mathbb{R}^2 $, $ \epsilon > 0 $ y $ \delta = \sqrt{\epsilon} $. Como $ \lvert f(x,y) \rvert \leq x^2 + y^2 $ se cumple que 

        $$ 0 \leq \lvert f(0,0) \rvert \leq 0^2 + 0^2 = 0 $$

        Por lo que $ f(0,0) = 0 $.

        Luego, si $ 0 < \lVert (x,y) - (0,0) \rVert < \delta $ entonces

        $$ \lVert (x,y) \rVert < \sqrt{\epsilon} $$

        $$ \Longrightarrow \sqrt{x^2 + y^2} < \sqrt{\epsilon} $$

        $$ \Longrightarrow x^2 + y^2 < \epsilon $$

        $$ \Longrightarrow \lvert f(x,y) \rvert \leq x^2 + y^2 < \epsilon $$

        $$ \Longrightarrow \lvert f(x,y) \rvert < \epsilon $$

        $$ \Longrightarrow \lvert f(x,y) - 0 \rvert < \epsilon $$

        $$ \Longrightarrow \lvert f(x,y) - f(0,0) \rvert < \epsilon $$

        Por lo tanto, $ f $ es continua en $ \overline{0} $. $ \blacksquare $

        %Tercer ejercicio
        \item \textbf{Demostración.}

        Como $ f $ es continua en $ \overline{x_0} $ se tiene que $ \forall \epsilon > 0 $, en particular para $ \epsilon = f(\overline{x_0}) $, existe $ \delta > 0 $ tal que si $ \overline{y} \in B_\delta (\overline{x_0}) $ entonces $ f(\overline{y}) \in B_\epsilon (f(\overline{x_0})) $. Así, $ \lvert f(\overline{x_0}) - f(\overline{y}) \rvert < \epsilon = f(\overline{x_0}) $ y de esto

        $$ -f(\overline{x_0} < f(\overline{x_0}) - f(\overline{y}) < f(\overline{x_0}) $$

        $$ \Longrightarrow f(\overline{x_0}) - f(\overline{y}) < f(\overline{x_0}) $$

        $$ \Longrightarrow - f(\overline{y}) < 0 $$

        $$ \therefore f(\overline{y}) > 0 $$. $ \blacksquare $

        %Cuarto ejercicio
        \item \textbf{Demostración.}

        Sea $ \overline{x_0} = (x_0,y_0) \in \mathbb{R}^2 $ un punto fijo. La derivada de $ f $ en $ \overline{x_0} $ es

        $$ D f(\overline{x_0}) = (1,1) $$ 
        
        Luego,
        \begin{align*}
            \lim_{(\overline{x} \to \overline{x_0}} \dfrac{\lvert f(\overline{x}) - (1(x - x_0) + 1(y - y_0) + f(\overline{x_0})) \rvert}{\lVert \overline{x} - \overline{x_0} \rVert} &= \lim_{(\overline{x} \to \overline{x_0}} \dfrac{\lvert 2 + x + y - (x - x_0 + y - y_0 + 2 + x_0 + y_0) \rvert}{\lVert \overline{x} - \overline{x_0} \rVert} \\
            &= \lim_{(\overline{x} \to \overline{x_0}} \dfrac{\lvert 2 + x + y - (x + y + 2) \rvert}{\lVert \overline{x} - \overline{x_0} \rVert} \\
            &= \lim_{(\overline{x} \to \overline{x_0}} \dfrac{\lvert 0 \rvert}{\lVert \overline{x} - \overline{x_0} \rVert} \\
            &= 0
        \end{align*}
        Por lo tanto, $ f $ es diferenciable en cualquier punto.

        %Quinto ejercicio
        \item 

    \end{enumerate}
\end{document}