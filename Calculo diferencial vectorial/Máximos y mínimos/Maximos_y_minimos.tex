\documentclass[fleqn, 12pt]{article}
\usepackage[a4paper, margin = 21mm]{geometry}
\usepackage[spanish]{babel}
\usepackage{parskip}
\usepackage{amsmath, amsfonts, amsthm}
\usepackage{enumerate}
\usepackage{graphicx}
\usepackage[p,osf]{scholax}
% T1 and textcomp are loaded by package. Change that here, if you want
% load sans and typewriter packages here, if needed
%\usepackage{amsmath,amsthm} must be loaded before newtxmath
% amssymb should not be loaded
\usepackage[scaled=1.075,ncf,vvarbb]{newtxmath}% need to scale up math package
% vvarbb selects the STIX version of blackboard bold.
\newcommand{\derivadaparcial}[2]{\dfrac{\partial {#1}}{\partial {#2}}}
\newcommand{\derivadaparcialn}[3]{\dfrac{\partial^{#3} {#1}}{\partial {#2}^{#3}}}
\newcommand{\derivadaparcialnd}[3]{\dfrac{\partial^{2} {#1}}{\partial {#3} \partial {#2}}}
\newcommand{\talque}{\; \middle | \;}

\begin{document}
    \begin{center}
        UNIVERSIDAD AUTÓNOMA DEL ESTADO DE MÉXICO \\
        FACULTAD DE CIENCIAS \\
        DEPARTEMENTO DE MATEMÁTICAS \\
        Cálculo Diferencial Vectorial \\
        Profesores: Dr. Félix Capulín Pérez \\
        Dr. Enrique Castañeda Alvarado \\
        Tarea: Máximos y mínimos
    \end{center}

    Nombre: Osmar Dominique Santana Reyes \hfill No. de cuenta: $ 2125197 $

    \textbf{Instrucciones:} Resuelve cada uno de los ejercicios, justifica cada respuesta.

    \begin{enumerate}[\bfseries Ejerc{i}c{i}o 1.]
%-------Ejercicio 1---------------------------------------------------------------------------------------------------
        \item Encontrar los valores máximos y mínimos locales, asi como los puntos silla de las funciones siguientes, después usar un software graficador en tres dimensiones y grafica las funciones teniendo en cuenta que se visualicen todos los aspectos importantes de la función correspondiente.
        
        \begin{enumerate}
%-----------Inciso a) del Ejercicio 1---------------------------------------------------------------------------------
            \item $ f(x,y) = 1 + 2xy - x^2 - y^2 $.
            
            \textbf{Solución.}

            Calculando los puntos críticos de la función:

            $ \derivadaparcial{f}{x} = 2y - 2x = 0 \quad $ y $ \quad \derivadaparcial{f}{y} = 2x - 2y = 0 $

            $ \Longrightarrow 2y - 2x = 0 \quad $ y $ \quad 2x - 2y = 0 $

            $ \Longrightarrow y = x $

            Asi, el conjunto $ A = \left\lbrace (x,y) \in \mathbb{R}^2 \talque x = y \right\rbrace $ contiene a todos los puntos críticos de la función $ f $.

            Luego, se sabe que

            $ 0 \leq (x - y)^2 = x^2 - 2xy + y^2 $

            $ \Longrightarrow 2xy - x^2 - y^2 \leq 0 $

            $ \Longrightarrow 1 + 2xy - x^2 - y^2 \leq 1 $

            Pero si $ (x,y) \in A $ entonces

            $ f(x,y) = 1 + 2xx - x^2 - x^2 = 1 $

            Por lo tanto, $ f $ alcanza su máximo absoluto en los puntos del conjunto $ A $.

            \begin{figure}[h]
                \centering
                \includegraphics[height = 5cm]{Primero.png}
            \end{figure}

%-----------Inciso b) del Ejercicio 1---------------------------------------------------------------------------------
            \item $ f(x,y) = xy - 2x - 2y $.
            
            \textbf{Solución.}

            Calculando los puntos críticos de la función:

            $ \derivadaparcial{f}{x} = y - 2 = 0 \quad $ y $ \quad \derivadaparcial{f}{y} = x - 2 = 0 $

            $ \Longrightarrow y - 2 = 0 \quad $ y $ \quad x - 2 = 0 $

            $ \Longrightarrow y = 2 \quad $ y $ \quad x = 2 $

            Asi, $ (2,2) $ es un punto crítico de $ f $.

            Si $ y = x $ entonces $ f(x,y) = x^2 - 4x $, el cual alcanza su mínimo en $ x = 2 $.

            Si $ y = 4 - x $ entonces $ f(x,y) = x(4-x) - 2x - 2(4-x) = 4x - x^2 - 8 $, el cual alcanza su máximo en $ x = 2 $.

            Por lo tanto, $ (2,2) $ es un punto silla de $ f $.

            \begin{figure}[h]
                \centering
                \includegraphics[height = 7cm]{Segundo.png}
            \end{figure}

%-----------Inciso c) del Ejercicio 1---------------------------------------------------------------------------------
            \item $ f(x,y) = \dfrac{x^2 y^2 - 8x + y}{xy} $.
            
            \textbf{Solución.}

            Calculando las primeras y segundas derivadas parciales:
            \begin{align*}
                \derivadaparcial{f}{x} &= \dfrac{(xy)(2x y^2 - 8) - (x^2 y^2 - 8x + y)(y)}{x^2 y^2} \\
                &= \dfrac{x(2x y^2 - 8) - (x^2 y^2 - 8x + y)}{x^2 y} \\
                &= \dfrac{2x^2 y^2 - 8x - x^2 y^2 + 8x - y}{x^2 y} \\
                &= \dfrac{x^2 y^2 - y}{x^2 y} \\
                &= \dfrac{x^2 y - 1}{x^2}
            \end{align*}
            \begin{align*}
                \derivadaparcial{f}{y} &= \dfrac{(xy)(2x^2 y + 1) - (x^2 y^2 - 8x + y)(x)}{x^2 y^2} \\
                &= \dfrac{y(2x^2 y + 1) - (x^2 y^2 - 8x + y)}{x y^2} \\
                &= \dfrac{2x^2 y^2 + y - x^2 y^2 + 8x - y}{x y^2} \\
                &= \dfrac{x^2 y^2 + 8x}{x y^2} \\
                &= \dfrac{x y^2 + 8}{y^2}
            \end{align*}
            \begin{align*}
                \derivadaparcialn{f}{x}{2} &= \dfrac{x^2(2xy) - (x^2 y - 1)(2x)}{x^4} \\
                &= \dfrac{2x^3 y - 2x^3 y + 2x}{x^4} \\
                &= \dfrac{2x^2 y - 2x^2 y + 2}{x^3} \\
                &= \dfrac{2}{x^3}
            \end{align*}
            \begin{align*}
                \derivadaparcialn{f}{y}{2} &= \dfrac{y^2(2xy) - (x y^2 + 8)(2y)}{y^4} \\
                &= \dfrac{y(2xy) - (x y^2 + 8)(2)}{y^3} \\
                &= \dfrac{2x y^2 - 2x y^2 - 16}{y^3} \\
                &= - \dfrac{16}{y^3}
            \end{align*}
            \begin{align*}
                \derivadaparcialnd{f}{x}{y} = 1 = \derivadaparcialnd{f}{y}{x}
            \end{align*}
           
            Calculando los puntos críticos de la función:

            $ \derivadaparcial{f}{x} = 0 \quad $ y $ \quad \derivadaparcial{f}{y} = 0 $

            $ \Longrightarrow \dfrac{x^2 y - 1}{x^2} = 0 \quad $ y $ \quad \dfrac{x y^2 + 8}{y^2} = 0 $

            De esta manera se tiene el siguiente sistema de ecuaciones
            \begin{align}
                x^2 y - 1 = 0 \label{eq:1c1} \\
                xy^2 + 8 = 0 \label{eq:1c2}
            \end{align}
            De (\ref{eq:1c1}) se tiene que $ y = \dfrac{1}{x^2} $. Sustituyendo esto en (\ref{eq:1c2}), se tiene que

            $ x \left( \dfrac{1}{x^2} \right)^2 + 8 = 0 $

            $ \Longrightarrow \dfrac{1}{x^3} = -8 $

            $ \Longrightarrow x^3 = -\dfrac{1}{8} $

            $ \Longrightarrow x = -\dfrac{1}{2} $

            Sustituyendo en (\ref{eq:1c1}): 

            $ \left( -\dfrac{1}{2} \right)^2 y - 1 = 0 $

            $ \Longrightarrow \dfrac{y}{4} = 1 $

            $ \Longrightarrow y = 4 $

            Así, $ \left( -\dfrac{1}{2}, 4 \right) $ es un punto crítico de $ f $.

            Luego, 
            \begin{align*}
                D \left( -\dfrac{1}{2}, 4 \right) &= \left( \derivadaparcialn{f}{x}{2} \left( -\dfrac{1}{2}, 4 \right) \right) \left( \derivadaparcialn{f}{y}{2} \left( -\dfrac{1}{2}, 4 \right) \right) - \left( \derivadaparcialnd{f}{x}{y} \left( -\dfrac{1}{2}, 4 \right) \right)^2 \\
                &= \left( \dfrac{2}{\left( -\frac{1}{2} \right)^3} \right) \left( - \dfrac{16}{4^3} \right) - 1^2 \\
                &= (-16) \left( - \dfrac{1}{4} \right) - 1 \\
                &= 4 - 1 \\
                &= 3 > 0
            \end{align*}
            Y como $ \derivadaparcialn{f}{x}{2} \left( -\dfrac{1}{2}, 4 \right) = -16 < 0 $ entonces $ \left( -\dfrac{1}{2}, 4 \right) $ es un punto de máximo local de $ f $.

            \begin{figure}[h]
                \centering
                \includegraphics[height = 5cm]{Tercero.png}
            \end{figure}

%-----------Inciso d) del Ejercicio 1---------------------------------------------------------------------------------
            \item $ f(x,y) = e^x \cos (y) $.
            
            \textbf{Solución.}

            Calculando las primeras y segundas derivadas parciales:
            \begin{align*}
                \derivadaparcial{f}{x} &=  e^x \cos (y) \qquad & \derivadaparcial{f}{y} &= -e^x \sen (y) \\
                \derivadaparcialn{f}{x}{2} &= e^x \cos (y) & \derivadaparcialn{f}{y}{2} &= -e^x \cos (y) \\
                \derivadaparcialnd{f}{x}{y} &= -e^x \sen (y) = \derivadaparcialnd{f}{y}{x}
            \end{align*}
            Calculando los puntos críticos de la función:

            $ \derivadaparcial{f}{x} = 0 \quad $ y $ \quad \derivadaparcial{f}{y} = 0 $

            $ \Longrightarrow e^x \cos (y) = 0 \quad $ y $ \quad -e^x \sen (y) = 0 $

            $ \Longrightarrow \cos (y) = 0 \quad $ y $ \quad - \sen (y) = 0 $

            $ \Longrightarrow y = k \, \dfrac{\pi}{2} $ con $ k \in \mathbb{N} $

            Asi, el conjunto $ B = \left\lbrace (x,y) \in \mathbb{R}^2 \talque y = k \, \dfrac{\pi}{2} \text{ con } k \in \mathbb{N} \cup \lbrace 0 \rbrace \right\rbrace $ contiene a todos los puntos críticos de la función $ f $.

            Luego, sea $ (x,y) \in B $ se tiene que $ y = k \, \dfrac{\pi}{2} \text{ con } k \in \mathbb{N} \cup \lbrace 0 \rbrace $. Así,

            $  D(x,y) = \left( \derivadaparcialn{f}{x}{2} (x,y) \right) \left( \derivadaparcialn{f}{y}{2} (x,y) \right) - \left( \derivadaparcialnd{f}{x}{y} (x,y) \right)^2 $

            \begin{itemize}
                \item  Si $ k = 0, 4, 8, 12, \ldots $ entonces
            
                $ D(x,y) = \left( e^x \right) \left( -e^x \right) = -e^{2x} < 0 $

                \item  Si $ k = 1, 5, 9, 13, \ldots $ entonces

                $  D(x,y) = - \left( -e^x \right)^2 < 0 $

                \item  Si $ k = 2, 6, 10, 14, \ldots $ entonces

                $  D(x,y) = \left( -e^{x} \right) \left( e^{x} \right) = -e^{2x} < 0 $

                \item  Si $ k = 3, 7, 11, 15, \ldots $ entonces

                $  D(x,y) = - \left( e^x \right)^2 < 0 $
            \end{itemize}

           Por lo tanto, los puntos del conjunto $ B $ son puntos silla de $ f $.
            
%-----------Inciso e) del Ejercicio 1---------------------------------------------------------------------------------
            \item $ f(x,y) = x \sen (y) $.
            
            \textbf{Solución.}

            Calculando los puntos críticos de la función:

            $ \derivadaparcial{f}{x} = \sen (y) = 0 \quad $ y $ \quad \derivadaparcial{f}{y} = x \cos (y) = 0 $

            $ \Longrightarrow \sen (y) = 0 \quad $ y $ \quad x \cos (y) = 0 $

            $ \Longrightarrow y = k \pi \text{ con } k \in \mathbb{N} \quad $ y $ \quad x \cos (y) = 0 $

            $ \Longrightarrow y = k \pi \text{ con } k \in \mathbb{N} \quad $ y $ \quad x \lvert 1 \rvert = 0 $

            $ \Longrightarrow y = k \pi \text{ con } k \in \mathbb{N} \quad $ y $ \quad x = 0 $

            Asi, el conjunto $ C = \left\lbrace (x,y) \in \mathbb{R}^2 \talque x = 0, y = k \pi \text{ con } k \in \mathbb{N} \right\rbrace $ contiene a todos los puntos críticos de la función $ f $.

            Ahora, si $ (x,y) \in C $ entonces $ f(x,y) = 0 $. Pero para $ (x,y) = \left( 1, \dfrac{\pi}{2} \right) $, $ f(x,y) = 1 $ y para $ (x,y) = \left( 1, \dfrac{\pi}{2} \right) $, $ f(x,y) = -1 $. Debido a esto, los puntos del conjunto $ C $ no son máximos ni mínimos locales, es decir, son puntos silla.
            
            \begin{figure}[h]
                \centering
                \includegraphics[height = 5cm]{Quinto.png}
            \end{figure}

        \end{enumerate}

%-------Ejercicio 2---------------------------------------------------------------------------------------------------
        \item Encontrar los valores máximo y mínimo absolutos de $ f $ en la región $ D $.
        
        \begin{enumerate}
%-----------Inciso a) del Ejercicio 2---------------------------------------------------------------------------------
            \item $ f(x,y) = x^2 + y^2 + x^2 y + 4, \, D = \left\lbrace (x,y) \in \mathbb{R}^2 : \lvert x \rvert \leq 1, \lvert y \rvert \leq 1 \right\rbrace $.
            
            \textbf{Solución.}

            Sea $ (x,y) \in D $ entonces $ \lvert x \rvert \leq 1 \quad $ y $ \quad \lvert y \rvert \leq 1 $

            $ \Longrightarrow -1 \leq x \leq 1 \quad $ y $ \quad -1 \leq y \leq 1 $

            $ \Longrightarrow 0 \leq x^2 \leq 1 \quad $ y $ \quad 0 \leq y^2 \leq 1 $

            $ \Longrightarrow 0 \leq x^2 \leq 1 $, $ \quad 0 \leq y^2 \leq 1 \quad $ y $ \quad 0 \leq x^2 y \leq 1 $

            $ \Longrightarrow 4 \leq x^2 + y^2 + x^2 y + 4 \leq 1 + 1 + 1 + 4 = 7 $

            De esta forma, $ f $ está acotada entre 4 y 7. Luego, notemos que 
            
            $ f(0,0) = 0^2 + 0^2 + 0^2 (0) + 4 = 4 \quad $ y 
            
            $ f(1,1) = 1^2 + 1^2 + 1^2 (1) + 4 = 7 = (-1)^2 + 1^2 + (-1)^2 (1) + 4 = f(-1,1) $.

            Por lo tanto, $ f $ alcanza su mínimo absoluto en $ (0,0) $ y su máximo absoluto en $ (-1,1) $ y $ (1,1) $.

%-----------Inciso b) del Ejercicio 2---------------------------------------------------------------------------------
            \item $ f(x,y) = 1 + xy - x - y, \, D $ es la región acotada por la parábola $ y = x^2 $ y la recta $ y = 4 $.

            % \textbf{Solución.}

            % Sea $ (x,y) \in D $. Como $ D $ está acotada por $ y = x^2 $ y $ y \leq 4 $ (pues $ D $ también está acotada por la recta $ y = 4 $), se tiene que $ 0 \leq y \leq 4 $. Así, $ D = \left\lbrace (x,y) \in \mathbb{R}^2 \talque 0 \leq y \leq 4 \text{\; y \;} x^2 - y \leq 0 \right\rbrace $. 

            % Luego, sean $ g(x,y) = y \, $ y $ \, h(x,y) = x^2 - y $ se sabe que
            
            % $ \nabla f = \lambda \nabla g + \alpha \nabla h $

            % $ \Longrightarrow (y - 1, x - 1) = \lambda (0, 1) + \alpha (2x, -1) $
            % \begin{align}
            %     \Longrightarrow \; & y - 1 = 2 \alpha x \label{eq:2b1} \\
            %     & x - 1 = \lambda - \alpha \label{eq:2b2} \\
            %     & 0 \leq y \leq 4 \label{eq:2b3} \\
            %     & x^2 - y \leq 0 \label{eq:2b4}
            % \end{align}


%-----------Inciso c) del Ejercicio 2---------------------------------------------------------------------------------
            \item $ f(x,y) = 2x^3 + y^4, \, D = \left\lbrace (x,y) \in \mathbb{R}^2 : x^2 + y^2 \leq 1 \right\rbrace $.

            % \textbf{Solución.}
            
            % $ \nabla f = \lambda \nabla (x^2 + y^2) $

            % $ \Longrightarrow (6x^2, 4y^3) = \lambda (2x, 2y) $
            % \begin{align}
            %     \Longrightarrow & 6x^2 = 2 \lambda x \\
            %     & 4y^3 = 2 \lambda y \\
            %     & x^2 + y^2 \leq 1
            % \end{align}
            
            
        \end{enumerate}
%-------Ejercicio 3---------------------------------------------------------------------------------------------------
        \item Encontrar el punto del plano $ 2x - y + z = 1 $ que sea más cercano al punto $ (-4,1,3) $.

        \textbf{Solución.}
        
        Sea $ (x,y,z) $ un punto del plano $ 2x - y + z = 1 $ entonces la distancia entre este punto y el punto $ (-4,1,3) $ esta dada por 

        $ \sqrt{(x + 4)^2 + (y - 1)^2 + (z - 3)^2} $

        Luego, de la ecuación del plano $ 2x - y + z = 1 $ se tiene que $ z = 1 + y - 2x $. Sustituyendo esto en la expresión anterior se tiene que

        $ \sqrt{(x + 4)^2 + (y - 1)^2 + (1 + y - 2x - 3)^2} = \sqrt{(x + 4)^2 + (y - 1)^2 + (y - 2x - 2)^2} $

        Como la raíz cuadrada es una función creciente entonces basta con encontrar el mínimo absoluto de $ (x + 4)^2 + (y - 1)^2 + (y - 2x - 2)^2 $.

        Así, sea $ g(x,y) = (x + 4)^2 + (y - 1)^2 + (y - 2x - 2)^2 $. Calculando las primeras y segundas derivadas parciales:

        \begin{align*}
            & \derivadaparcial{g}{x} =  2(x + 4) - 4(y - 2x - 2) = 2x + 8 - 4y + 8x + 8 = 10x - 4y + 16 \\
            & \derivadaparcial{g}{y} = 2(y - 1) + 2(y - 2x - 2) = 2y - 2 + 2y - 4x - 4 = -4x + 4y - 6 \\
            & \derivadaparcialn{g}{x}{2} = 10 \qquad \derivadaparcialn{g}{y}{2} = 4 \\
            & \derivadaparcialnd{g}{x}{y} = -4 = \derivadaparcialnd{g}{y}{x}
        \end{align*}
        Calculando los puntos críticos de la función:

        $ \derivadaparcial{g}{x} = 0 \quad $ y $ \quad \derivadaparcial{g}{y} = 0 $

        $ \Longrightarrow 10x - 4y + 16 = 0 \quad $ y $ \quad -4x + 4y - 6 = 0 $

        De esta manera, se forma el siguiente sistema de ecuaciones
        \begin{align}
            10x - 4y = -16 \label{eq:41} \\
            -4x + 4y = 6 \label{eq:42}
        \end{align}
        Sumando (\ref{eq:41}) y (\ref{eq:42}): $ 6x = -10 \Longrightarrow x = -\dfrac{5}{3} $. Sustituyendo en (\ref{eq:42}) se tiene que $ -4 \left( -\dfrac{5}{3} \right) + 4y = 6 \Longrightarrow 4y = 6 - \dfrac{20}{3} \Longrightarrow y = -\dfrac{2}{12} = -\dfrac{1}{6} $

        Así, $ \left( -\dfrac{5}{3}, -\dfrac{1}{6} \right) $ es un punto crítico de $ g $. Luego,
        \begin{align*}
            D \left( -\dfrac{5}{3}, -\dfrac{1}{6} \right) &= \left( \derivadaparcialn{g}{x}{2} \left( -\dfrac{5}{3}, -\dfrac{1}{6} \right) \right) \left( \derivadaparcialn{g}{y}{2} \left( -\dfrac{5}{3}, -\dfrac{1}{6} \right) \right) - \left( \derivadaparcialnd{g}{x}{y} \left( -\dfrac{5}{3}, -\dfrac{1}{6} \right) \right)^2 \\
            &= (10)(4) - (-4)^2 \\
            &= 40 - 16 \\
            &= 24 > 0
        \end{align*}
        Y como $ \derivadaparcialn{g}{x}{2} \left( -\dfrac{5}{3}, -\dfrac{1}{6} \right) = 10 > 0 $, $ g $ tiene un punto mínimo en $ \left( -\dfrac{5}{3}, -\dfrac{1}{6} \right) $. Después,
        \begin{align*}
            \sqrt{\left( -\dfrac{5}{3} + 4 \right)^2 + \left( -\dfrac{1}{6} - 1 \right)^2 + \left( -\dfrac{1}{6} - 2 \left( -\dfrac{5}{3} \right) - 2 \right)^2} &= \sqrt{\left( \dfrac{7}{3} \right)^2 + \left( -\dfrac{7}{6} \right)^2 + \left( \dfrac{7}{6} \right)^2} \\
            &= \sqrt{\dfrac{49}{9} + \dfrac{49}{36} + \dfrac{49}{36}} \\
            &= \sqrt{\dfrac{294}{36}} \\
            &= \dfrac{7 \sqrt{6}}{6}
        \end{align*}
        Por lo tanto la distancia mínima entre el punto del plano $ 2x - y + z = 1 $ al punto $ (-4,1,3) $ es $ \dfrac{7 \sqrt{6}}{6} $.

%-------Ejercicio 4---------------------------------------------------------------------------------------------------
        \item Encontrar el punto de la superficie $ x^2 y^2 z = 1 $ que sea más cercano al origen.
        
        \textbf{Solución.}

        Sea $ (x,y,z) $ un punto de la superficie $ 2x^2 y^2 z = 1 $ entonces la distancia entre este punto y el origen esta dada por 

        $ \sqrt{x^2 + y^2 + z^2} $

        Luego, de la ecuación de la superficie $ 2x^2 y^2 z = 1 $ se tiene que $ z = \dfrac{1}{2x^2 y^2} $. Sustituyendo esto en la expresión anterior se tiene que

        $ \sqrt{x^2 + y^2 + \left( \dfrac{1}{2x^2 y^2} \right)^2} = \sqrt{x^2 + y^2 + \dfrac{1}{4x^4 y^4}} $

        Como la raíz cuadrada es una función creciente entonces basta con encontrar el mínimo absoluto de $ x^2 + y^2 + \dfrac{1}{4x^4 y^4} $.

        Así, sea $ h(x,y) = x^2 + y^2 + \dfrac{1}{4x^4 y^4} $. Calculando las primeras y segundas derivadas parciales:
        \begin{align*}
            & \derivadaparcial{h}{x} = 2x - \dfrac{16x^3 y^4}{(4x^4 y^4)^2} = 2x - \dfrac{1}{x^5 y^4} \\
            & \derivadaparcial{h}{y} = 2y - \dfrac{16x^4 y^3}{(4x^4 y^4)^2} = 2y - \dfrac{1}{x^4 y^5} \\
            & \derivadaparcialn{h}{x}{2} = 2 + \dfrac{5x^4 y^4}{(x^5 y^4)^2} = 2 + \dfrac{5}{x^6 y^4} \\
            & \derivadaparcialn{h}{y}{2} = 2 + \dfrac{5x^4 y^4}{(x^4 y^5)^2} = 2 + \dfrac{5}{x^4 y^6} \\
            & \derivadaparcialnd{h}{x}{y} = \dfrac{4x^5 y^3}{(x^5 y^4)^2} = \dfrac{4}{x^5 y^5} = \dfrac{4x^3 y^5}{(x^4 y^5)^2} = \derivadaparcialnd{h}{y}{x}
        \end{align*}
        Buscando los puntos críticos de la función:

        $ \derivadaparcial{h}{x} = 0 \quad $ y $ \quad \derivadaparcial{h}{y} = 0 $

        $ \Longrightarrow 2x - \dfrac{1}{x^5 y^4} = 0 \quad $ y $ \quad 2y - \dfrac{1}{x^4 y^5} = 0 $

        $ \Longrightarrow \dfrac{2x^6 y^4 - 1}{x^5 y^4} = 0 \quad $ y $ \quad \dfrac{2x^4 y^6 - 1}{x^4 y^5} = 0 $

        $ \Longrightarrow 2x^6 y^4 - 1 = 0 \quad $ y $ \quad 2x^4 y^6 - 1 = 0 $

        $ \Longrightarrow 2x^6 y^4 = 1 \quad $ y $ \quad 2x^4 y^6 = 1 $

        $ \Longrightarrow 2x^6 y^4 = 2x^4 y^6 $

        $ \Longrightarrow x^2 = y^2 $

        $ \Longrightarrow x = y \quad $ o $ \quad x = -y $

        Si $ x = y $ entonces $ 2x^6 y^4 = 2x^{10} = 1 \Longrightarrow x = \dfrac{1}{\sqrt[10]{2}} = y $.

        Si $ x = -y $ entonces $ 2(-y)^6 y^4 = -2y^{10} = 1 $. Pero no hay algún número real que cumpla esta igualdad.

        De esta forma, $ \left( \dfrac{1}{\sqrt[10]{2}}, \dfrac{1}{\sqrt[10]{2}} \right) $ es un punto crítico de $ h $. Luego,
        \begin{align*}
            D \left( \dfrac{1}{\sqrt[10]{2}}, \dfrac{1}{\sqrt[10]{2}} \right) &= \left( \derivadaparcialn{h}{x}{2} \left( \dfrac{1}{\sqrt[10]{2}}, \dfrac{1}{\sqrt[10]{2}} \right) \right) \left( \derivadaparcialn{h}{y}{2} \left( \dfrac{1}{\sqrt[10]{2}}, \dfrac{1}{\sqrt[10]{2}} \right) \right) - \\
            & \quad \left( \derivadaparcialnd{h}{x}{y} \left( \dfrac{1}{\sqrt[10]{2}}, \dfrac{1}{\sqrt[10]{2}} \right) \right)^2 \\
            &= (12)(12) - (8)^2 \\
            &= 144 - 64 \\
            &= 80 > 0
        \end{align*}
        Y como $ \derivadaparcialn{h}{x}{2} \left( \dfrac{1}{\sqrt[10]{2}}, \dfrac{1}{\sqrt[10]{2}} \right) = 12 > 0 $, $ h $ tiene un punto mínimo en $ \left( \dfrac{1}{\sqrt[10]{2}}, \dfrac{1}{\sqrt[10]{2}} \right) $. Después,
        \begin{align*}
            \sqrt{\left( \dfrac{1}{\sqrt[10]{2}} \right)^2 + \left( \dfrac{1}{\sqrt[10]{2}} \right)^2 + \dfrac{1}{4 \left( \frac{1}{\sqrt[10]{2}} \right)^4 \left( \frac{1}{\sqrt[10]{2}} \right)^4}} = \left( \sqrt[10]{2} \right)^4
        \end{align*}
        Por lo tanto la distancia mínima entre un punto de la superficie al origen es $ \left( \sqrt[10]{2} \right)^4 $.
    \end{enumerate}
\end{document}
