\documentclass[fleqn, 12pt]{article}
\usepackage[a4paper, margin = 21mm]{geometry}
\usepackage[spanish]{babel}
\usepackage{parskip}
\usepackage{amsmath, amsfonts, amsthm}
\usepackage{enumerate}
\usepackage{graphicx}
\usepackage[p,osf]{scholax}
\usepackage[scaled=1.075,ncf,vvarbb]{newtxmath}

\newcommand{\derivadaparcial}[2]{\dfrac{\partial {#1}}{\partial {#2}}}
\newcommand{\derivadaparcialn}[3]{\dfrac{\partial^{#3} {#1}}{\partial {#2}^{#3}}}
\newcommand{\derivadaparcialnd}[3]{\dfrac{\partial^{2} {#1}}{\partial {#3} \partial {#2}}}
\newcommand{\talque}{\; \middle | \;}

\begin{document}
\begin{center}
    UNIVERSIDAD AUTÓNOMA DEL ESTADO DE MÉXICO \\
    FACULTAD DE CIENCIAS \\
    DEPARTAMENTO DE MATEMÁTICAS \\
    Cálculo Diferencial Vectorial \\
    Profesores: Dr. Félix Capulín Pérez \\
    Dr. Enrique Castañeda Alvarado \\
    Tarea: Multiplicadores de Lagrange
\end{center}

Nombre: Osmar Dominique Santana Reyes \hfill No. de cuenta: $ 2125197 $

\textbf{Instrucciones:} Resuelve cada uno de los ejercicios, justifica cada respuesta.

\begin{list}{\bfseries Ejercicio}{ \addtolength{\itemindent}{-1mm}%
    \addtolength{\labelsep}{-1mm}%
    \addtolength{\leftmargin}{-1cm}% 
    \addtolength{\labelwidth}{-1cm} }
%---Ejercicio 1----------------------------------------------------------------------------------------------------------
    \item $ \mathbf{1.} $ Utiliza multiplicadores de Lagrange para hallar los valores máximos y mínimos de la función, sujeta a la(s) restricción(es) dada(s). 
    
    \begin{enumerate}[a)]
%-------Inciso a) del Ejercicio 1----------------------------------------------------------------------------------------
        \item $ f(x,y) = x^2 y $; \; $ x^2 + 2y^2 = 6 $
        
        \textbf{Solución.}

        $ \nabla f = \lambda \nabla (x^2 + 2y^2) $

        $ \Longrightarrow (2xy, x^2) = \lambda (2x, 4y) $
        \begin{align}
            \Longrightarrow \; & 2xy = 2 \lambda x \label{eq:1a1} \\
            & x^2 = 4 \lambda y \label{eq:1a2} \\
            & x^2 + 2y^2 = 6 \label{eq:1a3}
        \end{align}
        \begin{itemize}
            \item Si $ x = 0 $ entonces, por (\ref{eq:1a3}) $ 2y^2 = 6 \Longrightarrow y = \sqrt{3} $ o $ y = - \sqrt{3} $.
            \item Luego, si $ x \neq 0 $ entonces $ y = \lambda $, por (\ref{eq:1a1}). Sustituyendo esto en (\ref{eq:1a2}) se tiene que $ x^2 = 4y^2 \Longrightarrow \dfrac{x^2}{2} = 2y^2 $. De lo anterior, y por \ref{eq:1a3}, se da que $ x^2 + \dfrac{x^2}{2} = \dfrac{3x^2}{2} = 6 \Longrightarrow x^2 = 4 \Longrightarrow x = 2 $ o $ x = -2 $. Como $ x^2 = 4, \, \lambda = y $ y por (\ref{eq:1a2}) se obtiene que $ 4 = 4y^2 \Longrightarrow y = 1 $ o $ y = -1 $.
        \end{itemize}
        Así, se tienen los puntos $ \left( 0,\sqrt{3} \right), \, \left( 0,- \sqrt{3} \right), \, (2,1), \, (-2,1), \, (2,-1) $ y $ (-2,-1) $. Evaluandolos en la función se tiene que

        $ f \left( 0, \sqrt{3} \right) = 0^2 \cdot \sqrt{3} = 0 = 0^2 \cdot - \sqrt{3} = f \left( 0, - \sqrt{3} \right) $

        $ f(2,1) = 2^2 \cdot 1 = 4 = (-2)^2 \cdot 1 = f(-2,1) $

        $ f(2,-1) = 2^2 \cdot -1 = -4 = (-2)^2 \cdot -1 = f(-2,-1) $

        Por lo tanto, el valor máximo de $ f $ es $ 4 $ y su valor mínimo es $ -4 $.
%-------Inciso b) del Ejercicio 1----------------------------------------------------------------------------------------
        \item $ f(x,y,z) = x^2 + y^2 + z^2 $; \; $ x^4 + y^4 + z^4 = 1 $
        
        \textbf{Solución.}

        $ \nabla f = \lambda \nabla (x^4 + y^4 + z^4) $

        $ \Longrightarrow (2x, 2y, 2z) = \lambda (4x^3, 4y^3, 4z^3) $

        $ \Longrightarrow (x, y, z) = \lambda (2x^3, 2y^3, 2z^3) $
        \begin{align}
            \Longrightarrow \; & x = 2 \lambda x^3 \label{eq:1b1} \\
            & y = 2 \lambda y^3 \label{eq:1b2} \\
            & z = 2 \lambda z^3 \label{eq:1b3} \\
            & x^4 + y^4 + z^4 = 1 \label{eq:1b4}
        \end{align}
        $ \lambda \neq 0 $, pues de lo contrario, de (\ref{eq:1b1}), (\ref{eq:1b2}) y (\ref{eq:1b3}), se tendría que $ x = y = z = 0 $ y no se cumpliría (\ref{eq:1b4}). Luego, considerando los siguientes casos:

        \begin{itemize}
            \item Si $ x = 0 $.
            \begin{itemize}
                \item Si $ y = 0 $ entonces de (\ref{eq:1b4}) \; $ z^4 = 1 \Longrightarrow z = 1 \quad $ o $ \quad z = -1 $.
                
                \item Si $ y \neq 0 $.
                \begin{itemize}
                    \item Si $ z = 0 $ entonces de (\ref{eq:1b4}) \; $ y^4 = 1 \Longrightarrow y = 1 \quad $ o $ \quad y = -1 $.
                    \item Si $ z \neq 0 $ entonces de (\ref{eq:1b2}) $ \lambda = \dfrac{1}{2y^2} $ y de (\ref{eq:1b3}) $ \lambda = \dfrac{1}{2z^2} $. Así, $ \dfrac{1}{2y^2} = \dfrac{1}{2z^2} \Longrightarrow z^2 = y^2 \Longrightarrow z^4 = y^4 $. Sustituyendo en (\ref{eq:1b4}): $ y^4 + y^4 = 1 \Longrightarrow 2y^4 = 1 \Longrightarrow y^4 = \dfrac{1}{2} \Longrightarrow y = - \dfrac{1}{\sqrt[4]{2}} \quad $ o $ \quad y = \dfrac{1}{\sqrt[4]{2}} $. De esta manera, $ z^4 = \dfrac{1}{2} \Longrightarrow z = - \dfrac{1}{\sqrt[4]{2}} \quad $ o $ \quad z = \dfrac{1}{\sqrt[4]{2}} $.
                \end{itemize}
            \end{itemize}

            \item Si $ x \neq 0 $.
            \begin{itemize}
                \item Si $ y = 0 $.
                \begin{itemize}
                    \item Si $ z = 0 $ entonces de (\ref{eq:1b4}) \; $ x^4 = 1 \Longrightarrow x = 1 \quad $ o $ \quad x = -1 $.
                    \item Si $ z \neq 0 $ entonces de (\ref{eq:1b1}) $ \lambda = \dfrac{1}{2x^2} $ y de (\ref{eq:1b3}) $ \lambda = \dfrac{1}{2z^2} $. Así, $ \dfrac{1}{2x^2} = \dfrac{1}{2z^2} \Longrightarrow z^2 = x^2 \Longrightarrow z^4 = x^4 $. Sustituyendo en (\ref{eq:1b4}): $ x^4 + x^4 = 1 \Longrightarrow 2x^4 = 1 \Longrightarrow x^4 = \dfrac{1}{2} \Longrightarrow x = - \dfrac{1}{\sqrt[4]{2}} \quad $ o $ \quad x = \dfrac{1}{\sqrt[4]{2}} $. De esta manera, $ z^4 = \dfrac{1}{2} \Longrightarrow z = - \dfrac{1}{\sqrt[4]{2}} \quad $ o $ \quad z = \dfrac{1}{\sqrt[4]{2}} $.
                \end{itemize}

                \item Si $ y \neq 0 $.
                \begin{itemize}
                    \item Si $ z = 0 $ entonces de (\ref{eq:1b1}) $ \lambda = \dfrac{1}{2x^2} $ y de (\ref{eq:1b2}) $ \lambda = \dfrac{1}{2y^2} $. Así, $ \dfrac{1}{2x^2} = \dfrac{1}{2y^2} \Longrightarrow y^2 = x^2 \Longrightarrow y^4 = x^4 $. Sustituyendo en (\ref{eq:1b4}): $ x^4 + x^4 = 1 \Longrightarrow 2x^4 = 1 \Longrightarrow x^4 = \dfrac{1}{2} \Longrightarrow x = - \dfrac{1}{\sqrt[4]{2}} \quad $ o $ \quad x = \dfrac{1}{\sqrt[4]{2}} $. De esta manera, $ y^4 = \dfrac{1}{2} \Longrightarrow y = - \dfrac{1}{\sqrt[4]{2}} \quad $ o $ \quad y = \dfrac{1}{\sqrt[4]{2}} $.
                    \item Si $ z \neq 0 $ entonces de (\ref{eq:1b1}), (\ref{eq:1b2}) y (\ref{eq:1b3}) se tiene que 
                    
                    $ \lambda = \dfrac{1}{2x^2} = \dfrac{1}{2y^2} = \dfrac{1}{2z^2} $

                    $ \Longrightarrow x^2 = y^2 = z^2 $

                    $ \Longrightarrow x^4 = y^4 = z^4 $

                    Sustituyendo en (\ref{eq:1b4}):

                    $ x^4 + x^4 + x^4 = 1 $

                    $ \Longrightarrow 3x^4 = 1 $

                    $ \Longrightarrow x^4 = \dfrac{1}{3} $

                    $ \Longrightarrow x = \dfrac{1}{\sqrt[4]{3}} \quad $ o $ \quad x = - \dfrac{1}{\sqrt[4]{3}} $

                    De esta forma, $ y = \dfrac{1}{\sqrt[4]{3}} \quad $ o $ \quad y = - \dfrac{1}{\sqrt[4]{3}} \quad $ y $ \quad z = \dfrac{1}{\sqrt[4]{3}} \quad $ o $ \quad z = - \dfrac{1}{\sqrt[4]{3}} $.
                \end{itemize}
            \end{itemize}   
        \end{itemize}

        De todo lo anterior, se obtienen los siguientes puntos: 
                    
        $ (0,0,1), (0,0,-1), (0,1,0), (0,-1,0), \left( 0, \dfrac{1}{\sqrt[4]{2}}, \dfrac{1}{\sqrt[4]{2}} \right), \left( 0, \dfrac{1}{\sqrt[4]{2}}, - \dfrac{1}{\sqrt[4]{2}} \right), \left( 0, -\dfrac{1}{\sqrt[4]{2}}, \dfrac{1}{\sqrt[4]{2}} \right), $ \\
        $ \left( 0, -\dfrac{1}{\sqrt[4]{2}}, -\dfrac{1}{\sqrt[4]{2}} \right), (1,0,0), (-1,0,0), \left( \dfrac{1}{\sqrt[4]{2}}, 0, \dfrac{1}{\sqrt[4]{2}} \right), \left( \dfrac{1}{\sqrt[4]{2}}, 0, - \dfrac{1}{\sqrt[4]{2}} \right), \left( - \dfrac{1}{\sqrt[4]{2}}, 0, \dfrac{1}{\sqrt[4]{2}} \right), $ \\
        $ \left( - \dfrac{1}{\sqrt[4]{2}}, 0, - \dfrac{1}{\sqrt[4]{2}} \right), \left( \dfrac{1}{\sqrt[4]{2}}, \dfrac{1}{\sqrt[4]{2}}, 0 \right), \left( \dfrac{1}{\sqrt[4]{2}}, - \dfrac{1}{\sqrt[4]{2}}, 0 \right), \left( - \dfrac{1}{\sqrt[4]{2}}, \dfrac{1}{\sqrt[4]{2}}, 0 \right), \left( - \dfrac{1}{\sqrt[4]{2}}, - \dfrac{1}{\sqrt[4]{2}}, 0 \right), $ \\
        $ \left( \dfrac{1}{\sqrt[4]{3}}, \dfrac{1}{\sqrt[4]{3}}, \dfrac{1}{\sqrt[4]{3}} \right), \left( \dfrac{1}{\sqrt[4]{3}}, \dfrac{1}{\sqrt[4]{3}}, - \dfrac{1}{\sqrt[4]{3}} \right), \left( \dfrac{1}{\sqrt[4]{3}}, - \dfrac{1}{\sqrt[4]{3}}, \dfrac{1}{\sqrt[4]{3}} \right), \left( \dfrac{1}{\sqrt[4]{3}}, - \dfrac{1}{\sqrt[4]{3}}, - \dfrac{1}{\sqrt[4]{3}} \right), $ \\       
        $ \left( - \dfrac{1}{\sqrt[4]{3}}, \dfrac{1}{\sqrt[4]{3}}, \dfrac{1}{\sqrt[4]{3}} \right), \left( - \dfrac{1}{\sqrt[4]{3}}, \dfrac{1}{\sqrt[4]{3}}, - \dfrac{1}{\sqrt[4]{3}} \right), \left( - \dfrac{1}{\sqrt[4]{3}}, - \dfrac{1}{\sqrt[4]{3}}, \dfrac{1}{\sqrt[4]{3}} \right), \left( - \dfrac{1}{\sqrt[4]{3}}, - \dfrac{1}{\sqrt[4]{3}}, - \dfrac{1}{\sqrt[4]{3}} \right) $.

        Si $ (x,y,z) \in \left\lbrace (0,0,1), (0,0,-1), (0,1,0), (0,-1,0), (1,0,0), (-1,0,0) \right\rbrace $ entonces \\ $ f(x,y,z) = 1 $.

        Si $ (x,y,z) \in \left\lbrace \left( 0, \dfrac{1}{\sqrt[4]{2}}, \dfrac{1}{\sqrt[4]{2}} \right), \left( 0, \dfrac{1}{\sqrt[4]{2}}, - \dfrac{1}{\sqrt[4]{2}} \right), \left( 0, -\dfrac{1}{\sqrt[4]{2}}, \dfrac{1}{\sqrt[4]{2}} \right), \left( 0, -\dfrac{1}{\sqrt[4]{2}}, -\dfrac{1}{\sqrt[4]{2}} \right), \right. $ \\
        $ \left( \dfrac{1}{\sqrt[4]{2}}, 0, \dfrac{1}{\sqrt[4]{2}} \right), \left( \dfrac{1}{\sqrt[4]{2}}, 0, - \dfrac{1}{\sqrt[4]{2}} \right), \left( - \dfrac{1}{\sqrt[4]{2}}, 0, \dfrac{1}{\sqrt[4]{2}} \right), \left( - \dfrac{1}{\sqrt[4]{2}}, 0, - \dfrac{1}{\sqrt[4]{2}} \right), \left( \dfrac{1}{\sqrt[4]{2}}, \dfrac{1}{\sqrt[4]{2}}, 0 \right), $ \\   
        $ \left. \left( \dfrac{1}{\sqrt[4]{2}}, - \dfrac{1}{\sqrt[4]{2}}, 0 \right), \left( - \dfrac{1}{\sqrt[4]{2}}, \dfrac{1}{\sqrt[4]{2}}, 0 \right), \left( - \dfrac{1}{\sqrt[4]{2}}, - \dfrac{1}{\sqrt[4]{2}}, 0 \right) \right\rbrace $ entonces $ f(x,y,z) = \sqrt{2} $.

        Si $ (x,y,z) \in \left\lbrace \left( \dfrac{1}{\sqrt[4]{3}}, \dfrac{1}{\sqrt[4]{3}}, \dfrac{1}{\sqrt[4]{3}} \right), \left( \dfrac{1}{\sqrt[4]{3}}, \dfrac{1}{\sqrt[4]{3}}, - \dfrac{1}{\sqrt[4]{3}} \right), \left( \dfrac{1}{\sqrt[4]{3}}, - \dfrac{1}{\sqrt[4]{3}}, \dfrac{1}{\sqrt[4]{3}} \right), \right. $ \\
        $ \left( \dfrac{1}{\sqrt[4]{3}}, - \dfrac{1}{\sqrt[4]{3}}, - \dfrac{1}{\sqrt[4]{3}} \right), \left( - \dfrac{1}{\sqrt[4]{3}}, \dfrac{1}{\sqrt[4]{3}}, \dfrac{1}{\sqrt[4]{3}} \right), \left( - \dfrac{1}{\sqrt[4]{3}}, \dfrac{1}{\sqrt[4]{3}}, - \dfrac{1}{\sqrt[4]{3}} \right), \left( - \dfrac{1}{\sqrt[4]{3}}, - \dfrac{1}{\sqrt[4]{3}}, \dfrac{1}{\sqrt[4]{3}} \right), $ \\
        $ \left. \left( - \dfrac{1}{\sqrt[4]{3}}, - \dfrac{1}{\sqrt[4]{3}}, - \dfrac{1}{\sqrt[4]{3}} \right) \right\rbrace $ entonces $ f(x,y,z) = \sqrt{3} $.

        Por lo tanto, el valor máximo de $ f $ es $ \sqrt{3} $ y su valor mínimo es $ 1 $.

%-------Inciso c) del Ejercicio 1----------------------------------------------------------------------------------------
        \centering
        \includegraphics[width = 1.0\linewidth]{Ejercicio c1.jpg}
       
        \includegraphics[width = 1.0\linewidth]{Ejercicio c2.jpg}
    
%-------Inciso d) del Ejercicio 1----------------------------------------------------------------------------------------
        \includegraphics[width = 1.0\linewidth]{Ejercicio d1.jpg}
       
        \includegraphics[width = 1.0\linewidth]{Ejercicio d2.jpg}
       
%-------Inciso e) del Ejercicio 1----------------------------------------------------------------------------------------
        \includegraphics[width = 1.0\linewidth]{Ejercicio e1.jpg}
       
        \includegraphics[width = 1.0\linewidth]{Ejercicio e2.jpg}
       
        \includegraphics[width = 1.0\linewidth]{Ejercicio e3.jpg}
    \end{enumerate}

    \normalfont
%---Ejercicio 2----------------------------------------------------------------------------------------------------------
    \item $ \mathbf{2.} $ Encontrar los valores máximo y mínimo absolutos de $ f $ en la región descrita por la desigualdad:
    
    \begin{enumerate}[a)]
%-------Inciso a) del Ejercicio 2----------------------------------------------------------------------------------------
        \item $ f(x,y) = 2x^2 + 3y^2 - 4x - 5 $; \; $ x^2 + y^2 \leq 16 $
        
        \textbf{Solución.}

%-------Inciso b) del Ejercicio 2----------------------------------------------------------------------------------------
        \item $ f(x,y) = e^{-xy} $; \; $ x^2 + 4y^2 \leq 1 $
        
        \textbf{Solución.}

        $ \nabla f = \lambda \nabla (x^2 + 4y^2) $

        $ \Longrightarrow (-y e^{-xy}, -x e^{-xy}) = \lambda (2x, 8y) $
        \begin{align}
            \Longrightarrow \; & -y e^{-xy} = 2 \lambda x \label{eq:2b1} \\
            & -x e^{-xy} = 8 \lambda y \label{eq:2b2} \\
            &  x^2 + 4y^2 \leq 1 \label{eq:2b3}
        \end{align}

    \end{enumerate}

%---Ejercicio 3----------------------------------------------------------------------------------------------------------
    \item $ \mathbf{3.} $ Usa multiplicadores de Lagrange para demostrar que el rectángulo con máxima área, que tiene un perímetro dado $ p $ es un cuadrado.
    
    \textbf{Solución.}

    Sean $ x, y $ la base y la altura de un rectángulo, respectivamente; $ P(x,y) = 2x + 2y = p $ su perímetro y $ A(x,y) = xy $ su área. Usando multiplicadores de Lagrange se tiene que

    $ \nabla A = \lambda \nabla P $

    $ \Longrightarrow (y,x) = \lambda (2,2) $
    \begin{align}
        \Longrightarrow & y = 2 \lambda \label{eq:31} \\
        & x = 2 \lambda \label{eq:32} \\
        & 2x + 2y = p \label{eq:33}
    \end{align}
    Sustituyendo (\ref{eq:31}) y (\ref{eq:32}) en (\ref{eq:33}): $ 2(2 \lambda) + 2(2 \lambda) = p \Longrightarrow 8 \lambda = p \Longrightarrow \lambda = \dfrac{p}{8} $. Así, de (\ref{eq:31}) y (\ref{eq:32}) se obtiene que $ y = \dfrac{p}{4} $ y $ x = \dfrac{p}{4} $.

    Por lo tanto, el rectángulo alcanza su máxima área cuando la longitud de su base y altura son iguales, es decir, cuando es un cuadrado cuyos lados tienen una medida de un cuarto de su perímetro.

%---Ejercicio 4----------------------------------------------------------------------------------------------------------
    \item $ \mathbf{4.} $ Encuentra los volúmenes máximo y mínimo de una caja rectangular cuya superficie tiene un área de $ 1500 \text{ cm}^2 $ y para la cual la longitud total de las aristas es de $ 200 $ cm.
    
    \textbf{Solución.}

    Sean $ x,y,z $ el largo, el ancho y la altura de la caja, respectivamente; $ V(x,y,z) = xyz $ el volumen de la caja, $ A(x,y,z) = 2xy + 2yz + 2xz = 1500 $ su área superficial y $ L(x,y,z) = 4x + 4y + 4z = 200 $ la longitud total de las aristas. Usando multiplicadores de Lagrange se tiene que

    $ \nabla V = \alpha \nabla A + \lambda \nabla L $

    $ \Longrightarrow (yz, xz, xy) = \alpha (2y + 2z, 2x + 2z, 2y + 2x) + \lambda (4,4,4) $
    \begin{align}
        \Longrightarrow & yz = 2 \alpha y + 2 \alpha z + 4 \lambda \label{eq:41} \\
        & xz = 2 \alpha x + 2 \alpha z + 4 \lambda \label{eq:42} \\
        & xy = 2 \alpha y + 2 \alpha x + 4 \lambda \label{eq:43} \\
        & 2xy + 2yz + 2xz = 1500 \Longrightarrow xy + yz + xz = 750 \label{eq:44} \\
        & 4x + 4y + 4z = 200 \Longrightarrow x + y + z = 50 \label{eq:45}
    \end{align}



%---Ejercicio 5----------------------------------------------------------------------------------------------------------
    \item $ \mathbf{5.} $ El plano $ x + y + 2z = 2 $ intersecta al paraboloide $ z = x^2 + y^2 $ en una elipse. Encontrar los puntos de esta elipse que estan más cerca y más lejos del origen.
    
%---Ejercicio 6----------------------------------------------------------------------------------------------------------
    \item $ \mathbf{6.} $ 
    
    \begin{enumerate}[a)]
%-------Inciso a) del Ejercicio 6----------------------------------------------------------------------------------------
        \item Encontrar el máximo valor de 
        $$ f(x_1, x_2, \ldots , x_n) = \sqrt[n]{\displaystyle x_1 x_2 \cdots x_n} $$
        dado que $ x_1, x_2, \ldots , x_n $ son números positivos y que $ x_1 + x_2 + \cdots + x_n = c $ donde $ c $ es una constante.


%-------Inciso b) del Ejercicio 6----------------------------------------------------------------------------------------
        \item Deducir del inciso a) que si $ x_1, x_2, \ldots , x_n $ son números positivos, entonces 
        $$ \sqrt[n]{\displaystyle x_1 x_2 \cdots x_n} = \dfrac{x_1 + x_2 + \cdots + x_n}{n} $$
        Note que esta desigualdad dice que la media geométrica de $ n $ números no es más grande que la media geométrica de los números. ¿Bajo qué circunstancias estas dos medias son iguales entre sí?

    \end{enumerate}
    
%---Ejercicio 7----------------------------------------------------------------------------------------------------------
    \item $ \mathbf{7.} $ 
    
    \begin{enumerate}
%-------Inciso a) del Ejercicio 7----------------------------------------------------------------------------------------
        \item Maximizar 
        $$ \sum_{i = 1}^{n} x_i y_i $$
        sujeta a las restricciones

        $ \displaystyle \sum_{i = 1}^{n} x_i^2 = 1 \quad $ y $ \displaystyle \quad \sum_{i = 1}^{n} y_i^2 = 1 $


%-------Inciso b) del Ejercicio 7----------------------------------------------------------------------------------------
        \item Haga 
            
        $ x_i = \dfrac{a_i}{\sqrt{\displaystyle \sum_{i = 1}^{n} a_i^2}} \quad $ y $ \quad x_i = \dfrac{b_i}{\sqrt{\displaystyle \sum_{i = 1}^{n} b_i^2}} $

        para demostrar que 
        $$ \sum_{i = 1}^{n} a_i b_i \leq \sqrt{\displaystyle \sum_{i = 1}^{n} a_i^2} \sqrt{\displaystyle \sum_{i = 1}^{n} b_i^2} $$
        para cualesquiera números $ a_1, a_2, \ldots , a_n, b_1, b_2, \ldots , b_n $. Esta desigualdad se conoce como la desigualdad de Cauchy-Schwarz.
    \end{enumerate}
\end{list}
\end{document}