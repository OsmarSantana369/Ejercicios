\documentclass[12pt]{article}
\usepackage[spanish]{babel}
\usepackage[margin = 21mm]{geometry}
\usepackage{amsmath, amssymb, amsfonts}
\usepackage{parskip}
\usepackage{multicol}
\usepackage{graphicx}

\usepackage[proportional,scaled=1]{erewhon}
\usepackage[erewhon,vvarbb,bigdelims]{newtxmath}
\usepackage[T1]{fontenc}
\renewcommand*\oldstylenums[1]{\textosf{#1}}

\expandafter\def\expandafter\normalsize\expandafter{%
    \setlength\abovedisplayskip{-9pt}%
    \setlength\belowdisplayskip{5pt}%
}

\begin{document}
	\textbf{4.2 Teoría del muestreo aleatorio estratificado.}

	Sea $ \Omega $ una población finita de $ N $ elementos y $ \left\lbrace \Omega_1, \Omega_2, \ldots, \Omega_m \right\rbrace $ una partición de $ \Omega $. A cada $ \Omega_i = \left\lbrace y_{i1}, y_{i2}, \ldots, y_{i N_i} \right\rbrace $ de tamaño $ N_i $, con $ i = 1, \ldots, m $, se le llama \textbf{estrato de la población}.

    Luego, para cada $ i = 1, \ldots, m $, sea $ \mathcal{S}_i $ una muestra aleatoria obtenida del estrato $ \Omega_i $ de tamaño $ n_i $. Las muestras obtenidas de cada estrato se juntan para formar una muestra de la población total, $ \Omega $.
\end{document}